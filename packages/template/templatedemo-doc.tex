\documentclass{ltxdoc}
\EnableCrossrefs
\CodelineIndex
\RecordChanges
%
\usepackage{cmap} 
\usepackage[T1]{fontenc} % T1 Schrift Encoding
\usepackage{lmodern}
%
%\RequirePackage{listings}
\RequirePackage[table]{xcolor}
\RequirePackage{enumitem}
\RequirePackage{tabu}

%\colorlet{lstcolorStringLatex}{green!40!black!100}
\colorlet{lstcolorCommentLatex}{green!50!black!100}
\definecolor{lstcolorKeywordLatex}{rgb}{0,0.47,0.80}

% define useless command for checking the
% existens of this style
\newcommand{\lstStyleLaTeX}{\relax}
% define style
\lstdefinestyle{lstStyleLaTeX}{%
   ,style=lstStyleBase
%%% colors
   ,stringstyle=\color{lstcolorStringLatex}%
   ,keywordstyle=\color{lstcolorKeywordLatex}%
   ,commentstyle=\color{lstcolorCommentLatex}%
   ,% backgroundcolor=\color{codebackcolor}%
%%% Frames
   ,frame=single%
   ,%frameround=tttt%
   ,%framesep = 10pt%
   ,%framerule = 0pt%
   ,rulecolor = \color{black}%
%%% language
   ,language = [LaTeX]TeX%
%%% commands
   % LaTeX programming
   ,moretexcs={setlength,usepackage,newcommand,renewcommand,providecommand,RequirePackage,SelectInputMappings,ifpdftex,ifpdfoutput,AtBeginEnvironment,ProvidesPackage},
   % other commands
   ,moretexcs={maketitle,text,includegraphics,chapter,section,subsection,
     subsubsection,paragraph,textmu,enquote,blockquote,ding,mathds,ifcsdef,Bra,Ket,Braket,subcaption,lettrine,mdfsetup,captionof,listoffigures,listoftables,tableofcontents,appendix}
   % tables
   ,moretexcs={newcolumntype,rowfont,taburowcolors,rowcolor,rowcolors,bottomrule,
     toprule,midrule,}
   % hyperref
   ,moretexcs={hypersetup}
   % glossaries
   ,moretexcs={gls,printglossary,glsadd,newglossaryentry,newacronym}
   % Koma
   ,moretexcs={mainmatter,frontmatter,geometry,KOMAoptions,setkomafont,addtokomafont}
   % SI, unit
   ,moretexcs={si,SI,sisetup,unit,unitfrac,micro}
   % biblatex package
   ,moretexcs={newblock,ExecuteBibliographyOptions,addbibresource}
   % math packages
   ,moretexcs={operatorname,frac,sfrac,dfrac,DeclareMathOperator,mathcal,underset}
   % demo package
   ,moretexcs={democodefile,package,cs,command,env,DemoError,PrintDemo}  
   % tablestyles
   ,moretexcs={theadstart,tbody,tsubheadstart,tsubhead,tend}
   % code section package
   ,moretexcs={DefineTemplateSection,SetTemplateSection,BeginTemplateSection,
     EndTemplateSection}
   % template tools package
   ,moretexcs={IfDefined,IfUndefined,IfElseDefined,IfElseUndefined,IfMultDefined,IfNotDraft,IfNotDraftElse,IfDraft,IfPackageLoaded,IfElsePackageLoaded,IfPackageNotLoaded,IfPackagesLoaded,IfPackagesNotLoaded,ExecuteAfterPackage,ExecuteBeforePackage,IfTikzLibraryLoaded,IfColumntypeDefined,IfColumntypesDefined,IfColorDefined,IfColorsDefined,IfMathVersionDefined,SetTemplateDefinition,UseDefinition,IfFileExists,iflanguage}
   % tablestyles
   ,moretexcs={setuptablefontsize,tablefontsize,setuptablestyle,tablestyle,  setuptablecolor,tablecolor,disablealternatecolors,   tablealtcolored,tbegin,tbody,tend,thead, theadstart,tsubheadstart,tsubhead,theadrow,tsubheadrow,resettablestyle,theadend,tsubheadend,tableitemize,PreserveBackslash}
   % todonotes
   ,moretexcs={todo,missingfigure}
   % listings
   ,moretexcs={lstloadlanguages,lstdefinestyle,lstset}
   % index
   ,moretexcs={indexsetup}
   % glossaries
   ,moretexcs={newglossarystyle,glossarystyle,deftranslation,newglossary}
   % tikz
   ,moretexcs={usetikzlibrary}
   % color
   ,moretexcs={definecolor,colorlet}
   % caption
   ,moretexcs={captionsetup,DeclareCaptionStyle}
   % floatrow
   ,moretexcs={floatsetup}
   % doc.sty
   ,moretexcs={EnableCrossrefs,DisableCrossrefs,PageIndex,CodelineIndex,CodelineNumbered}   
}
% \ifcsdef{addmoretexcs}{%
% LaTeX programming
\addmoretexcs[LaTeX]{setlength,usepackage,newcommand,renewcommand,providecommand,RequirePackage,SelectInputMappings,ifpdftex,ifpdfoutput,AtBeginEnvironment,ProvidesPackage}
% other commands
\addmoretexcs[LaTeX]{maketitle,text,includegraphics,chapter,section,subsection,
subsubsection,paragraph,textmu,enquote,blockquote,ding,mathds,ifcsdef,Bra,Ket,Braket,subcaption,lettrine,mdfsetup,captionof,listoffigures,listoftables,tableofcontents,appendix,url}
% tables
\addmoretexcs[LaTeX]{newcolumntype,rowfont,taburowcolors,rowcolor,rowcolors,bottomrule,toprule,midrule}
% hyperref
\addmoretexcs[LaTeX]{hypersetup}
% glossaries
\addmoretexcs[LaTeX]{gls,printglossary,glsadd,newglossaryentry,newacronym}
% Koma
\addmoretexcs[LaTeX]{mainmatter,frontmatter,geometry,KOMAoptions,setkomafont,addtokomafont}
% SI, unit
\addmoretexcs[LaTeX]{si,SI,sisetup,unit,unitfrac,micro}
% biblatex package
\addmoretexcs[LaTeX]{newblock,ExecuteBibliographyOptions,addbibresource}
% math packages
\addmoretexcs[LaTeX]{operatorname,frac,sfrac,dfrac,DeclareMathOperator,mathcal,underset}
% demo package
\addmoretexcs[LaTeX]{democodefile,package,cs,command,env,DemoError,PrintDemo}  
% tablestyles
\addmoretexcs[LaTeX]{theadstart,tbody,tsubheadstart,tsubhead,tend}
% code section package
\addmoretexcs[LaTeX]{DefineCodeSection,SetCodeSection,BeginCodeSection,EndCodeSection}
% template tools package
\addmoretexcs[LaTeX]{IfDefined,IfUndefined,IfElseDefined,IfElseUndefined,IfMultDefined,IfNotDraft,IfNotDraftElse,IfDraft,IfPackageLoaded,IfElsePackageLoaded,IfPackageNotLoaded,IfPackagesLoaded,IfPackagesNotLoaded,ExecuteAfterPackage,ExecuteBeforePackage,IfTikzLibraryLoaded,IfColumntypeDefined,IfColumntypesDefined,IfColorDefined,IfColorsDefined,IfMathVersionDefined,SetTemplateDefinition,UseDefinition,IfFileExists,iflanguage}
% tablestyles
\addmoretexcs[LaTeX]{setuptablefontsize,tablefontsize,setuptablestyle,tablestyle,setuptablecolor,tablecolor,disablealternatecolors,tablealtcolored,tbegin,tbody,tend,thead,theadstart,tsubheadstart,tsubhead,theadrow,tsubheadrow,resettablestyle,theadend,tsubheadend,tableitemize,PreserveBackslash}
% todonotes
\addmoretexcs[LaTeX]{todo,missingfigure}
% listings
\addmoretexcs[LaTeX]{lstloadlanguages,lstdefinestyle,lstset}
% index
\addmoretexcs[LaTeX]{indexsetup}
% glossaries
\addmoretexcs[LaTeX]{newglossarystyle,glossarystyle,deftranslation,newglossary}
% tikz
\addmoretexcs[LaTeX]{usetikzlibrary}
% color
\addmoretexcs[LaTeX]{definecolor,colorlet}
% caption
\addmoretexcs[LaTeX]{captionsetup,DeclareCaptionStyle}
% floatrow
\addmoretexcs[LaTeX]{floatsetup}
% doc.sty
\addmoretexcs[LaTeX]{EnableCrossrefs,DisableCrossrefs,PageIndex,CodelineIndex,CodelineNumbered}
% refereces
\addmoretexcs[LaTeX]{cref,Cref,vref,eqnref,figref,tabref,secref,chapref}
%
}{} % end of \ifcsdef

\lstloadlanguages{[LaTeX]TeX}
%
\makeatletter
\@ifpackageloaded{hypdoc}{}
   % load hyperref only if package 
   % hypdoc is not loaded, which
   % loads hyperref itself
   {\usepackage[
     ,backref=page%
     ,pagebackref=false%
     ,hyperindex=true%
     ,hyperfootnotes=false%
     ,bookmarks=true%
     ,pdfpagelabels=true%
   ]{hyperref}}   
\makeatother
%
\usepackage[]{bookmark}
%
\definecolor{pdfanchorcolor}{named}{black}
\definecolor{pdfmenucolor}{named}{red}
\definecolor{pdfruncolor}{named}{cyan}
\definecolor{pdfurlcolor}{rgb}{0,0,0.6}
\definecolor{pdffilecolor}{rgb}{0.7,0,0}
\definecolor{pdflinkcolor}{rgb}{0,0,0.6}
\definecolor{pdfcitecolor}{rgb}{0,0,0.6}
%
\hypersetup{
   ,draft=false, % all hypertext options are turned off
   ,final=true   % all hypertext options are turned on
   ,debug=false  % extra diagnostic messages are printed in the log file
   ,hypertexnames=true % use guessable names for links
   ,naturalnames=false % use LATEX-computed names for links
   ,setpagesize=true   % sets page size by special driver commands
   ,raiselinks=true    % forces commands to reflect the real height of the link 
   ,breaklinks=true    % Allows link text to break across lines
   ,pageanchor=true    % Determines whether every page is given an implicit
   ,plainpages=false   % Forces page anchors to be named by the arabic
   ,linktocpage=true   % make page number, not text, be link on TOC, LOF and LOT
   ,colorlinks=true    % Colors the text of links and anchors.
   ,linkcolor  =pdflinkcolor   % Color for normal internal links.
   ,anchorcolor=pdfanchorcolor % Color for anchor text.
   ,citecolor  =pdfcitecolor   % Color for bibliographical citations in text.
   ,filecolor  =pdffilecolor   % Color for URLs which open local files.
   ,menucolor  =pdfmenucolor   % Color for Acrobat menu items.
   ,runcolor   =pdfruncolor    % Color for run links (launch annotations).
   ,urlcolor   =pdfurlcolor    % color magenta Color for linked URLs.
   ,bookmarksopen=true     % If Acrobat bookmarks are requested, show them
   ,bookmarksopenlevel=2   % level (\maxdimen) to which bookmarks are open
   ,bookmarksnumbered=true %
   ,bookmarkstype=toc      %
   ,pdfpagemode=UseOutlines %
   ,pdfstartpage=1         % Determines on which page the PDF file is opened.
   ,pdfstartview=FitH      % Set the startup page view
   ,pdfremotestartview=Fit % Set the startup page view of remote PDF files
   ,pdfcenterwindow=false  %
   ,pdffitwindow=false     % resize document window to fit document size
   ,pdfnewwindow=false     % make links that open another PDF file 
   ,pdfdisplaydoctitle=true  % display document title instead of file name 
} % end: hypersetup
%

% color setup
% table colors 
\colorlet{tablebodycolor}{white!100}
\colorlet{tablerowcolor}{gray!10}
\colorlet{tableheadcolor}{gray!25}



\usepackage{templatedemo}
\usepackage{soul}


\newcommand{\package}[1]{\texttt{#1}}
\newcommand{\latex}{\LaTeX}

\listfiles
\begin{document}

\changes{0.1}{2011/12/15}{Initial version.}

\DoNotIndex{\newcommand,\newenvironment}

\providecommand*{\url}{\texttt}
\title{The \textsf{templatedemo} package}
\author{Matthias Pospiech \\ \url{matthias.pospiech@gmx.de}}
\date{0.1~from \filedate}

\maketitle
\begin{abstract}\noindent
This package provides some commands to demonstrate the result and the originating code sequence with a single occurance of the code within the document.
\end{abstract}
%\tableofcontents

\section{Introduction}
This package provides some commands to demonstrate the result of \latex{} code together with the code, which the result is based on. The difference to other packages (such as \package{showexpl}) is the support of verbatim material inside a conditional sequence. 
The consequence of this requirement is however that the commands of this package are more low level than origially intended.

The commands provided by this package are based on the packages \package{listings}, \package{mdframed} and \package{filecontents}. 


\section{Basic example}

Below is a principle example (using \cs{ifcsdef} from etoolbox):

\begin{lstlisting}[style=demostyle]
\begin{filecontents*}{\democodefile}
\so{letterspacing}, \\
\ul{underlining},   \\
\st{overstriking}   \\
and \hl{highlighting}. 
\end{filecontents*}

\ifcsdef{so}{%
\subsection*{package: soul}

Commands of package \demopackage{soul}:
\PrintCodeAndResultsParallel
}{% 
  \DemoError{Command \democs{so} of package% 
  \demopackage{soul} not available.
  Probably the package was not loaded.
  }
}%
\end{lstlisting}

which is shown as 

% ------------------------------------------------------------
\begin{filecontents*}{\democodefile}
\so{letterspacing}, \\
\ul{underlining},   \\
\st{overstriking}   \\
and \hl{highlighting}. 
\end{filecontents*}

\ifcsdef{so}{%
% ------------------------------------------------------------
\subsection*{package: soul}

Commands of package \demopackage{soul}:
\PrintCodeAndResultsParallel
}{}%



\section{Reference}

\subsection{Setup}


\begin{flushleft}
% Style changes
\small\rmfamily\renewcommand{\arraystretch}{1.4}  
% tabu
\begin{tabu} to 0.9\textwidth {X[1,l]>{a}X<{b}[1,l]X[2,l]}
\hline
\rowfont[l]{\bfseries}
\taburowcolors 1{tableheadcolor .. tableheadcolor}
command       &
default value & 
description  \\
\hline
\taburowcolors 2{tablebodycolor .. tablerowcolor}
\cs{democodefile} & 
\texttt{democode}  &
temporary file required for code and results printing. \\
% 
\cs{democodeprefix} & 
\texttt{Code: }  &
Prefix text for output of code in \\
% 
\hline
\end{tabu}
\end{flushleft}	


% better as table
%\newcommand{\democodeprefix}{Code: }
%\newcommand{\demoresultprefix}{\noindent Result:}

\subsection{Output commands}
%\newcommand{\democs}[1]{\democommand{#1}}
%\newcommand{\democommand}[1]{\texttt{\bs{}#1}}
%\newcommand{\demoenv}[1]{\texttt{#1}}
%\newcommand{\demopackage}[1]{\texttt{\itshape#1}}
%\DemoError
\subsection{Print code and result}
% Print code and result using the key-value syntax
%\newcommand{\PrintDemo}[1]{%

%  \ifnum\pdf@strcmp{\demo@style}{parallel}=0%
%    \PrintCodeAndResultsParallel%
%  \else\ifnum\pdf@strcmp{\demo@style}{stacked}=0%
%    \PrintCodeAndResultsStacked%
%  \else\ifnum\pdf@strcmp{\demo@style}{lines}=0%
%    \PrintCodeAndResultsStackedLines%
%  \else\ifnum\pdf@strcmp{\demo@style}{page}=0%
%    \PrintCodeAndResultsPage%
%  \else\ifnum\pdf@strcmp{\demo@style}{none}=0%

\subsection{Low level commands}

The printout of all examples shown in the previous section is based on the commands \cs{printlatexcode} and \cs{printlatexresult}, which are defined as:

\begin{lstlisting}[style=demostyle]
% Print code with prefix
\newcommand{\printlatexcode}{%
  \democodeprefix
  \lstinputlisting[style=demostyle,nolol=true]{\democodefile}%
}%
\end{lstlisting}

and 

\begin{lstlisting}[style=demostyle]
% Print result with standard box
\newcommand{\printlatexresult}{%
  \begin{latexresult}%
    \IfFileExists{\democodefile}{\input{\democodefile}}{}%
  \end{latexresult}%
}%
\end{lstlisting}

The environment \demoenv{latexresult} itself includes the command \cs{demoresultprefix} and a frame based on \package{mdframed}.

For all further commands used by the package and their imlementations please look at the code itsself. 
\end{document}
