\documentclass{ltxdoc}
\EnableCrossrefs
\CodelineIndex
\RecordChanges
%
\usepackage{cmap} 
\usepackage[T1]{fontenc} % T1 Schrift Encoding
\usepackage{lmodern}
%
\usepackage[table]{xcolor}
\usepackage{enumitem}
\usepackage{tabu}

%\colorlet{lstcolorStringLatex}{green!40!black!100}
\colorlet{lstcolorCommentLatex}{green!50!black!100}
\definecolor{lstcolorKeywordLatex}{rgb}{0,0.47,0.80}

% define useless command for checking the
% existens of this style
\newcommand{\lstStyleLaTeX}{\relax}
% define style
\lstdefinestyle{lstStyleLaTeX}{%
   ,style=lstStyleBase
%%% colors
   ,stringstyle=\color{lstcolorStringLatex}%
   ,keywordstyle=\color{lstcolorKeywordLatex}%
   ,commentstyle=\color{lstcolorCommentLatex}%
   ,% backgroundcolor=\color{codebackcolor}%
%%% Frames
   ,frame=single%
   ,%frameround=tttt%
   ,%framesep = 10pt%
   ,%framerule = 0pt%
   ,rulecolor = \color{black}%
%%% language
   ,language = [LaTeX]TeX%
%%% commands
   % LaTeX programming
   ,moretexcs={setlength,usepackage,newcommand,renewcommand,providecommand,RequirePackage,SelectInputMappings,ifpdftex,ifpdfoutput,AtBeginEnvironment,ProvidesPackage},
   % other commands
   ,moretexcs={maketitle,text,includegraphics,chapter,section,subsection,
     subsubsection,paragraph,textmu,enquote,blockquote,ding,mathds,ifcsdef,Bra,Ket,Braket,subcaption,lettrine,mdfsetup,captionof,listoffigures,listoftables,tableofcontents,appendix}
   % tables
   ,moretexcs={newcolumntype,rowfont,taburowcolors,rowcolor,rowcolors,bottomrule,
     toprule,midrule,}
   % hyperref
   ,moretexcs={hypersetup}
   % glossaries
   ,moretexcs={gls,printglossary,glsadd,newglossaryentry,newacronym}
   % Koma
   ,moretexcs={mainmatter,frontmatter,geometry,KOMAoptions,setkomafont,addtokomafont}
   % SI, unit
   ,moretexcs={si,SI,sisetup,unit,unitfrac,micro}
   % biblatex package
   ,moretexcs={newblock,ExecuteBibliographyOptions,addbibresource}
   % math packages
   ,moretexcs={operatorname,frac,sfrac,dfrac,DeclareMathOperator,mathcal,underset}
   % demo package
   ,moretexcs={democodefile,package,cs,command,env,DemoError,PrintDemo}  
   % tablestyles
   ,moretexcs={theadstart,tbody,tsubheadstart,tsubhead,tend}
   % code section package
   ,moretexcs={DefineTemplateSection,SetTemplateSection,BeginTemplateSection,
     EndTemplateSection}
   % template tools package
   ,moretexcs={IfDefined,IfUndefined,IfElseDefined,IfElseUndefined,IfMultDefined,IfNotDraft,IfNotDraftElse,IfDraft,IfPackageLoaded,IfElsePackageLoaded,IfPackageNotLoaded,IfPackagesLoaded,IfPackagesNotLoaded,ExecuteAfterPackage,ExecuteBeforePackage,IfTikzLibraryLoaded,IfColumntypeDefined,IfColumntypesDefined,IfColorDefined,IfColorsDefined,IfMathVersionDefined,SetTemplateDefinition,UseDefinition,IfFileExists,iflanguage}
   % tablestyles
   ,moretexcs={setuptablefontsize,tablefontsize,setuptablestyle,tablestyle,  setuptablecolor,tablecolor,disablealternatecolors,   tablealtcolored,tbegin,tbody,tend,thead, theadstart,tsubheadstart,tsubhead,theadrow,tsubheadrow,resettablestyle,theadend,tsubheadend,tableitemize,PreserveBackslash}
   % todonotes
   ,moretexcs={todo,missingfigure}
   % listings
   ,moretexcs={lstloadlanguages,lstdefinestyle,lstset}
   % index
   ,moretexcs={indexsetup}
   % glossaries
   ,moretexcs={newglossarystyle,glossarystyle,deftranslation,newglossary}
   % tikz
   ,moretexcs={usetikzlibrary}
   % color
   ,moretexcs={definecolor,colorlet}
   % caption
   ,moretexcs={captionsetup,DeclareCaptionStyle}
   % floatrow
   ,moretexcs={floatsetup}
   % doc.sty
   ,moretexcs={EnableCrossrefs,DisableCrossrefs,PageIndex,CodelineIndex,CodelineNumbered}   
}
% \ifcsdef{addmoretexcs}{%
% LaTeX programming
\addmoretexcs[LaTeX]{setlength,usepackage,newcommand,renewcommand,providecommand,RequirePackage,SelectInputMappings,ifpdftex,ifpdfoutput,AtBeginEnvironment,ProvidesPackage}
% other commands
\addmoretexcs[LaTeX]{maketitle,text,includegraphics,chapter,section,subsection,
subsubsection,paragraph,textmu,enquote,blockquote,ding,mathds,ifcsdef,Bra,Ket,Braket,subcaption,lettrine,mdfsetup,captionof,listoffigures,listoftables,tableofcontents,appendix,url}
% tables
\addmoretexcs[LaTeX]{newcolumntype,rowfont,taburowcolors,rowcolor,rowcolors,bottomrule,toprule,midrule}
% hyperref
\addmoretexcs[LaTeX]{hypersetup}
% glossaries
\addmoretexcs[LaTeX]{gls,printglossary,glsadd,newglossaryentry,newacronym}
% Koma
\addmoretexcs[LaTeX]{mainmatter,frontmatter,geometry,KOMAoptions,setkomafont,addtokomafont}
% SI, unit
\addmoretexcs[LaTeX]{si,SI,sisetup,unit,unitfrac,micro}
% biblatex package
\addmoretexcs[LaTeX]{newblock,ExecuteBibliographyOptions,addbibresource}
% math packages
\addmoretexcs[LaTeX]{operatorname,frac,sfrac,dfrac,DeclareMathOperator,mathcal,underset}
% demo package
\addmoretexcs[LaTeX]{democodefile,package,cs,command,env,DemoError,PrintDemo}  
% tablestyles
\addmoretexcs[LaTeX]{theadstart,tbody,tsubheadstart,tsubhead,tend}
% code section package
\addmoretexcs[LaTeX]{DefineCodeSection,SetCodeSection,BeginCodeSection,EndCodeSection}
% template tools package
\addmoretexcs[LaTeX]{IfDefined,IfUndefined,IfElseDefined,IfElseUndefined,IfMultDefined,IfNotDraft,IfNotDraftElse,IfDraft,IfPackageLoaded,IfElsePackageLoaded,IfPackageNotLoaded,IfPackagesLoaded,IfPackagesNotLoaded,ExecuteAfterPackage,ExecuteBeforePackage,IfTikzLibraryLoaded,IfColumntypeDefined,IfColumntypesDefined,IfColorDefined,IfColorsDefined,IfMathVersionDefined,SetTemplateDefinition,UseDefinition,IfFileExists,iflanguage}
% tablestyles
\addmoretexcs[LaTeX]{setuptablefontsize,tablefontsize,setuptablestyle,tablestyle,setuptablecolor,tablecolor,disablealternatecolors,tablealtcolored,tbegin,tbody,tend,thead,theadstart,tsubheadstart,tsubhead,theadrow,tsubheadrow,resettablestyle,theadend,tsubheadend,tableitemize,PreserveBackslash}
% todonotes
\addmoretexcs[LaTeX]{todo,missingfigure}
% listings
\addmoretexcs[LaTeX]{lstloadlanguages,lstdefinestyle,lstset}
% index
\addmoretexcs[LaTeX]{indexsetup}
% glossaries
\addmoretexcs[LaTeX]{newglossarystyle,glossarystyle,deftranslation,newglossary}
% tikz
\addmoretexcs[LaTeX]{usetikzlibrary}
% color
\addmoretexcs[LaTeX]{definecolor,colorlet}
% caption
\addmoretexcs[LaTeX]{captionsetup,DeclareCaptionStyle}
% floatrow
\addmoretexcs[LaTeX]{floatsetup}
% doc.sty
\addmoretexcs[LaTeX]{EnableCrossrefs,DisableCrossrefs,PageIndex,CodelineIndex,CodelineNumbered}
% refereces
\addmoretexcs[LaTeX]{cref,Cref,vref,eqnref,figref,tabref,secref,chapref}
%
}{} % end of \ifcsdef

\lstloadlanguages{[LaTeX]TeX}
%
\makeatletter
\@ifpackageloaded{hypdoc}{}
   % load hyperref only if package 
   % hypdoc is not loaded, which
   % loads hyperref itself
   {\usepackage[
     ,backref=page%
     ,pagebackref=false%
     ,hyperindex=true%
     ,hyperfootnotes=false%
     ,bookmarks=true%
     ,pdfpagelabels=true%
   ]{hyperref}}   
\makeatother
%
\usepackage[]{bookmark}
%
\definecolor{pdfanchorcolor}{named}{black}
\definecolor{pdfmenucolor}{named}{red}
\definecolor{pdfruncolor}{named}{cyan}
\definecolor{pdfurlcolor}{rgb}{0,0,0.6}
\definecolor{pdffilecolor}{rgb}{0.7,0,0}
\definecolor{pdflinkcolor}{rgb}{0,0,0.6}
\definecolor{pdfcitecolor}{rgb}{0,0,0.6}
%
\hypersetup{
   ,draft=false, % all hypertext options are turned off
   ,final=true   % all hypertext options are turned on
   ,debug=false  % extra diagnostic messages are printed in the log file
   ,hypertexnames=true % use guessable names for links
   ,naturalnames=false % use LATEX-computed names for links
   ,setpagesize=true   % sets page size by special driver commands
   ,raiselinks=true    % forces commands to reflect the real height of the link 
   ,breaklinks=true    % Allows link text to break across lines
   ,pageanchor=true    % Determines whether every page is given an implicit
   ,plainpages=false   % Forces page anchors to be named by the arabic
   ,linktocpage=true   % make page number, not text, be link on TOC, LOF and LOT
   ,colorlinks=true    % Colors the text of links and anchors.
   ,linkcolor  =pdflinkcolor   % Color for normal internal links.
   ,anchorcolor=pdfanchorcolor % Color for anchor text.
   ,citecolor  =pdfcitecolor   % Color for bibliographical citations in text.
   ,filecolor  =pdffilecolor   % Color for URLs which open local files.
   ,menucolor  =pdfmenucolor   % Color for Acrobat menu items.
   ,runcolor   =pdfruncolor    % Color for run links (launch annotations).
   ,urlcolor   =pdfurlcolor    % color magenta Color for linked URLs.
   ,bookmarksopen=true     % If Acrobat bookmarks are requested, show them
   ,bookmarksopenlevel=2   % level (\maxdimen) to which bookmarks are open
   ,bookmarksnumbered=true %
   ,bookmarkstype=toc      %
   ,pdfpagemode=UseOutlines %
   ,pdfstartpage=1         % Determines on which page the PDF file is opened.
   ,pdfstartview=FitH      % Set the startup page view
   ,pdfremotestartview=Fit % Set the startup page view of remote PDF files
   ,pdfcenterwindow=false  %
   ,pdffitwindow=false     % resize document window to fit document size
   ,pdfnewwindow=false     % make links that open another PDF file 
   ,pdfdisplaydoctitle=true  % display document title instead of file name 
} % end: hypersetup
%

% color setup
% table colors 
\colorlet{tablebodycolor}{white!100}
\colorlet{tablerowcolor}{gray!10}
\colorlet{tableheadcolor}{gray!25}

\setlength{\parindent}{0pt}
\setlength{\parskip}{0.5\baselineskip}

\usepackage{templatedemo}
\usepackage{soulutf8}

\newcommand{\package}[1]{\texttt{#1}}
\newcommand{\option}[1]{\texttt{#1}}
\newcommand{\parameter}[1]{\texttt{#1}}
\renewcommand\arg[1]{\meta{\normalfont\slshape#1}}
% 
\newcommand{\latex}{\LaTeX}
\newcommand{\Default}[1]{\par Default: \texttt{#1} \par}
\newcommand{\Example}[1]{\par Example: \texttt{#1} \par}

\newenvironment{Optionlist}{%
\begin{flushright}%
%  Style  changes
\small\renewcommand{\arraystretch}{1.4}%
%  tabu
\begin{tabu} to 1.0\linewidth  {>{\ttfamily}l<{\normalfont}X[1,l]}%
}{%
\end{tabu}%
\end{flushright}%
}

\setcounter{secnumdepth}{2}

\listfiles
\begin{document}

\changes{0.1}{2011/12/15}{Initial version.}

\DoNotIndex{\newcommand,\newenvironment}

\providecommand*{\url}{\texttt}
\title{The \textsf{templatedemo} package}
\author{Matthias Pospiech \\ \url{matthias.pospiech@gmx.de}}
\date{0.1~from \filedate}

\maketitle
\begin{abstract}\noindent
This package provides some commands to demonstrate the result and the originating code sequence with a single occurrence of the code within the document.
\end{abstract}
%\tableofcontents

\section{Introduction}
This package provides some commands to demonstrate the result of \latex{} code together with the code, which the result is based on. The difference to other packages (such as \package{showexpl}) is the support of verbatim material inside a conditional sequence. 
The consequence of this requirement is however that the commands of this package are more low level than originally intended.

The commands provided by this package are based on the packages \package{listings}, \package{mdframed} and \package{filecontents}. 


\section{Basic example}
\label{sec:example:basic}
Below is a principle example (using \cs{ifcsdef} from etoolbox),
which demonstrates some commands of the package \demopackage{soul}:

\begin{lstlisting}[style=demostyle]
\begin{filecontents*}{\democodefile}
\so{letterspacing}, \\
\ul{underlining},   \\
\st{overstriking}   \\
and \hl{highlighting}. 
\end{filecontents*}

\ifcsdef{so}{%
%
\PrintDemo{style=parallel}
%
}{% 
  \DemoError{Command \democs{so} of package% 
  \demopackage{soul} not available.
  Probably the package was not loaded.
  }
}%
\end{lstlisting}

which is shown as 

% ------------------------------------------------------------
\begin{filecontents*}{\democodefile}
\so{letterspacing}, \\
\ul{underlining},   \\
\st{overstriking}   \\
and \hl{highlighting}. 
\end{filecontents*}

\ifcsdef{so}{%
% ------------------------------------------------------------
\PrintDemo{style=parallel}
}{}%



\section{Usage}

\subsection{Define code}
This package requires the example code to be written to an external file. The filename is saved in the command sequence \cs{democodefile}. The output is done with the \demoenv{filecontents} environment. See following code for the usage.

\begin{lstlisting}[style=demostyle]
\begin{filecontents*}{\democodefile}
... code ...
\end{filecontents*}
\end{lstlisting}

The requirement for an external file originates from the problem that verbatim content cannot be saved in normal tex macros since the line breaks and white spaces get lost. Furthermore any other complex solution fails finally because saving such contend can not be included in conditional code statements%
\footnote{See discussion on: \url{http://tex.stackexchange.com/questions/29256/}}.

The content in the file defined by \cs{democodefile} is further read for the printing of the code and the corresponding output.

\subsection{Print code and result}

\DescribeMacro{\PrintDemo\{style=\arg{option}\}}%
is the macro for the output of code and result. The layout of both is defined with the style option. The following options are possible
%
\begin{Optionlist}
parallel &  code and result side by side.  \\
stacked  &  code and result with 100\,\% text width stacked.\\
lines    &  like \texttt{stacked}, but with lines on 
            top and bottom of the result instead of a surrounding box. \\
none     &  like \texttt{stacked}, but with no nothing around the result. \\
page     &  result on a single page.\\
\end{Optionlist}

You should use \option{parallel} for small examples and \option{stacked} otherwise.
The options \option{lines} and \option{none} is primarily for those cases where a surrounding box is disturbing or impossible. The latter occurs for example in cases where content is written across the text width boundaries. \option{page} is obviously for those cases where the output is very large or written to another page anyway.

\subsubsection{Examples}
The following code is used in the examples
\begin{lstlisting}[style=demostyle]
\begin{filecontents*}{\democodefile}
This code shows some basic math: $a^2 + b^2 = c^2$.
\end{filecontents*}
\end{lstlisting}

\begin{filecontents*}{\democodefile}
This code shows some basic math: $a^2 + b^2 = c^2$.
\end{filecontents*}

\begin{itemize}
\item \cmd{\PrintDemo\{style=parallel\}}
\PrintDemo{style=parallel}
%
\item \cmd{\PrintDemo\{style=stacked\}}
\PrintDemo{style=stacked}
%
\item \cmd{\PrintDemo\{style=lines\}}
\PrintDemo{style=lines}
%
\item \cmd{\PrintDemo\{style=none\}}
\PrintDemo{style=none}
\end{itemize}

The prefix text for code and result and the filename can be changed. The commands are introduced in section~\ref{sec:setup}. Commands for the output of content such as the name of of a package, a command, an environment or a generalized error message are introduced in section~\ref{sec:commands:output}.

\subsection{Setup}
\label{sec:setup}

The following commands define the name for the temporary file or the strings used for the printing of code or results. Use \cs{renewcommand} to change the definitions.

\DescribeMacro{\democodefile}%
Filename for the temporary file required for code and results printing. 
\Default{democode}
%
\DescribeMacro{\democodeprefix}%
Prefix text for output of code.
\Default{Code: }
%
\DescribeMacro{\demoresultprefix}%
Prefix text for output of the result. 
\Default{Result: }

\subsection{Output commands}
\label{sec:commands:output}

The following commands are provided for the user to print out and format commands, environments, packages and errors.

\DescribeMacro{\democs}%
shortcut for \cs{democommand}
\Example{\democs{foo}}
%
\DescribeMacro{\democommand}%
formats the given command sequence name. Note that a parameter
\parameter{foo} is printed as \cs{foo}.
\Example{\democommand{foo}}
%
\DescribeMacro{\demopackage}%
formats the given packagename.
\Example{\demopackage{foo}}
%
\DescribeMacro{\demoenv}%
formats the given environment name.
\Example{\demoenv{foo}}
%
\DescribeMacro{\DemoError}%
prints out the given error message
\Example{\DemoError{foo}}
%

\subsection{Low level commands}

The printout of all examples shown in the previous section is based on the commands \cs{printlatexcode} and \cs{printlatexresult}, which are defined as:

\begin{lstlisting}[style=demostyle]
% Print code with prefix
\newcommand{\printlatexcode}{%
  \democodeprefix
  \lstinputlisting[style=demostyle,nolol=true]{\democodefile}%
}%
\end{lstlisting}

and 

\begin{lstlisting}[style=demostyle]
% Print result with standard box
\newcommand{\printlatexresult}{%
  \begin{latexresult}%
    \IfFileExists{\democodefile}{\input{\democodefile}}{}%
  \end{latexresult}%
}%
\end{lstlisting}

The environment \demoenv{latexresult} itself includes the command \cs{demoresultprefix} and a frame based on \package{mdframed}.

For all further commands used by the package and their implementations please look at the code itself. 
\end{document}
