% \iffalse meta-comment
% 
% Copyright (C) 2012 by Matthias Pospiech <matthias@pospiech.eu>
% ---------------------------------------------------------------------------
% This work may be distributed and/or modified under the
% conditions of the LaTeX Project Public License, either version 1.3
% of this license or (at your option) any later version.
% The latest version of this license is in
%   http://www.latex-project.org/lppl.txt
% and version 1.3 or later is part of all distributions of LaTeX
% version 2005/12/01 or later.
%
% This work has the LPPL maintenance status `maintained'.
%
% The Current Maintainer of this work is Matthias Pospiech.
%
% This work consists of the files templatetools.dtx and templatetools.ins
% and the derived filebase templatetools.sty.
%
% \fi
%
% \iffalse
%<*driver>
\ProvidesFile{templatetools.dtx}
%</driver>
%<package>\NeedsTeXFormat{LaTeX2e}[1999/12/01]
%<package>\ProvidesPackage{templatetools}
%<*package>
    [<*date*> v0.1 Collection of conditional commands useful inside templates]
%</package>
%
%<*driver>
\documentclass{ltxdoc}
\usepackage{templatetools}[<*date*>]
\EnableCrossrefs
\CodelineIndex
\RecordChanges

%%% Font packages
\usepackage{cmap} 
\usepackage[T1]{fontenc} % T1 Schrift Encoding
\usepackage{lmodern}
%%% Additional packages
\usepackage{soulutf8}
\usepackage[table]{xcolor}
\usepackage{enumitem}
\usepackage{tabu}
% \usepackage[htt]{hyphenat}
%%% listings
\colorlet{lstcolorStringLatex}{green!40!black!100}
\colorlet{lstcolorCommentLatex}{green!50!black!100}
\definecolor{lstcolorKeywordLatex}{rgb}{0,0.47,0.80}

% define useless command for checking the
% existens of this style
\newcommand{\lstStyleLaTeX}{\relax}
% define style
\lstdefinestyle{lstStyleLaTeX}{%
   ,style=lstStyleBase
%%% colors
   ,stringstyle=\color{lstcolorStringLatex}%
   ,keywordstyle=\color{lstcolorKeywordLatex}%
   ,commentstyle=\color{lstcolorCommentLatex}%
   ,% backgroundcolor=\color{codebackcolor}%
%%% Frames
   ,frame=single%
   ,%frameround=tttt%
   ,%framesep = 10pt%
   ,%framerule = 0pt%
   ,rulecolor = \color{black}%
%%% language
   ,language = [LaTeX]TeX%
%%% commands
   % LaTeX programming
   ,moretexcs={setlength,usepackage,newcommand,renewcommand,providecommand,RequirePackage,SelectInputMappings,ifpdftex,ifpdfoutput,AtBeginEnvironment,ProvidesPackage},
   % other commands
   ,moretexcs={maketitle,text,includegraphics,chapter,section,subsection,
     subsubsection,paragraph,textmu,enquote,blockquote,ding,mathds,ifcsdef,Bra,Ket,Braket,subcaption,lettrine,mdfsetup,captionof,listoffigures,listoftables,tableofcontents,appendix}
   % tables
   ,moretexcs={newcolumntype,rowfont,taburowcolors,rowcolor,rowcolors,bottomrule,
     toprule,midrule,}
   % hyperref
   ,moretexcs={hypersetup}
   % glossaries
   ,moretexcs={gls,printglossary,glsadd,newglossaryentry,newacronym}
   % Koma
   ,moretexcs={mainmatter,frontmatter,geometry,KOMAoptions,setkomafont,addtokomafont}
   % SI, unit
   ,moretexcs={si,SI,sisetup,unit,unitfrac,micro}
   % biblatex package
   ,moretexcs={newblock,ExecuteBibliographyOptions,addbibresource}
   % math packages
   ,moretexcs={operatorname,frac,sfrac,dfrac,DeclareMathOperator,mathcal,underset}
   % demo package
   ,moretexcs={democodefile,package,cs,command,env,DemoError,PrintDemo}  
   % tablestyles
   ,moretexcs={theadstart,tbody,tsubheadstart,tsubhead,tend}
   % code section package
   ,moretexcs={DefineTemplateSection,SetTemplateSection,BeginTemplateSection,
     EndTemplateSection}
   % template tools package
   ,moretexcs={IfDefined,IfUndefined,IfElseDefined,IfElseUndefined,IfMultDefined,IfNotDraft,IfNotDraftElse,IfDraft,IfPackageLoaded,IfElsePackageLoaded,IfPackageNotLoaded,IfPackagesLoaded,IfPackagesNotLoaded,ExecuteAfterPackage,ExecuteBeforePackage,IfTikzLibraryLoaded,IfColumntypeDefined,IfColumntypesDefined,IfColorDefined,IfColorsDefined,IfMathVersionDefined,SetTemplateDefinition,UseDefinition,IfFileExists,iflanguage}
   % tablestyles
   ,moretexcs={setuptablefontsize,tablefontsize,setuptablestyle,tablestyle,  setuptablecolor,tablecolor,disablealternatecolors,   tablealtcolored,tbegin,tbody,tend,thead, theadstart,tsubheadstart,tsubhead,theadrow,tsubheadrow,resettablestyle,theadend,tsubheadend,tableitemize,PreserveBackslash}
   % todonotes
   ,moretexcs={todo,missingfigure}
   % listings
   ,moretexcs={lstloadlanguages,lstdefinestyle,lstset}
   % index
   ,moretexcs={indexsetup}
   % glossaries
   ,moretexcs={newglossarystyle,glossarystyle,deftranslation,newglossary}
   % tikz
   ,moretexcs={usetikzlibrary}
   % color
   ,moretexcs={definecolor,colorlet}
   % caption
   ,moretexcs={captionsetup,DeclareCaptionStyle}
   % floatrow
   ,moretexcs={floatsetup}
   % doc.sty
   ,moretexcs={EnableCrossrefs,DisableCrossrefs,PageIndex,CodelineIndex,CodelineNumbered}   
}
% \ifcsdef{addmoretexcs}{%
% LaTeX programming
\addmoretexcs[LaTeX]{setlength,usepackage,newcommand,renewcommand,providecommand,RequirePackage,SelectInputMappings,ifpdftex,ifpdfoutput,AtBeginEnvironment,ProvidesPackage}
% other commands
\addmoretexcs[LaTeX]{maketitle,text,includegraphics,chapter,section,subsection,
subsubsection,paragraph,textmu,enquote,blockquote,ding,mathds,ifcsdef,Bra,Ket,Braket,subcaption,lettrine,mdfsetup,captionof,listoffigures,listoftables,tableofcontents,appendix,url}
% tables
\addmoretexcs[LaTeX]{newcolumntype,rowfont,taburowcolors,rowcolor,rowcolors,bottomrule,toprule,midrule}
% hyperref
\addmoretexcs[LaTeX]{hypersetup}
% glossaries
\addmoretexcs[LaTeX]{gls,printglossary,glsadd,newglossaryentry,newacronym}
% Koma
\addmoretexcs[LaTeX]{mainmatter,frontmatter,geometry,KOMAoptions,setkomafont,addtokomafont}
% SI, unit
\addmoretexcs[LaTeX]{si,SI,sisetup,unit,unitfrac,micro}
% biblatex package
\addmoretexcs[LaTeX]{newblock,ExecuteBibliographyOptions,addbibresource}
% math packages
\addmoretexcs[LaTeX]{operatorname,frac,sfrac,dfrac,DeclareMathOperator,mathcal,underset}
% demo package
\addmoretexcs[LaTeX]{democodefile,package,cs,command,env,DemoError,PrintDemo}  
% tablestyles
\addmoretexcs[LaTeX]{theadstart,tbody,tsubheadstart,tsubhead,tend}
% code section package
\addmoretexcs[LaTeX]{DefineCodeSection,SetCodeSection,BeginCodeSection,EndCodeSection}
% template tools package
\addmoretexcs[LaTeX]{IfDefined,IfUndefined,IfElseDefined,IfElseUndefined,IfMultDefined,IfNotDraft,IfNotDraftElse,IfDraft,IfPackageLoaded,IfElsePackageLoaded,IfPackageNotLoaded,IfPackagesLoaded,IfPackagesNotLoaded,ExecuteAfterPackage,ExecuteBeforePackage,IfTikzLibraryLoaded,IfColumntypeDefined,IfColumntypesDefined,IfColorDefined,IfColorsDefined,IfMathVersionDefined,SetTemplateDefinition,UseDefinition,IfFileExists,iflanguage}
% tablestyles
\addmoretexcs[LaTeX]{setuptablefontsize,tablefontsize,setuptablestyle,tablestyle,setuptablecolor,tablecolor,disablealternatecolors,tablealtcolored,tbegin,tbody,tend,thead,theadstart,tsubheadstart,tsubhead,theadrow,tsubheadrow,resettablestyle,theadend,tsubheadend,tableitemize,PreserveBackslash}
% todonotes
\addmoretexcs[LaTeX]{todo,missingfigure}
% listings
\addmoretexcs[LaTeX]{lstloadlanguages,lstdefinestyle,lstset}
% index
\addmoretexcs[LaTeX]{indexsetup}
% glossaries
\addmoretexcs[LaTeX]{newglossarystyle,glossarystyle,deftranslation,newglossary}
% tikz
\addmoretexcs[LaTeX]{usetikzlibrary}
% color
\addmoretexcs[LaTeX]{definecolor,colorlet}
% caption
\addmoretexcs[LaTeX]{captionsetup,DeclareCaptionStyle}
% floatrow
\addmoretexcs[LaTeX]{floatsetup}
% doc.sty
\addmoretexcs[LaTeX]{EnableCrossrefs,DisableCrossrefs,PageIndex,CodelineIndex,CodelineNumbered}
% refereces
\addmoretexcs[LaTeX]{cref,Cref,vref,eqnref,figref,tabref,secref,chapref}
%
}{} % end of \ifcsdef

\lstloadlanguages{[LaTeX]TeX}
%%% hyperref
\makeatletter
\@ifpackageloaded{hypdoc}{}
   % load hyperref only if package 
   % hypdoc is not loaded, which
   % loads hyperref itself
   {\usepackage[
     ,backref=page%
     ,pagebackref=false%
     ,hyperindex=true%
     ,hyperfootnotes=false%
     ,bookmarks=true%
     ,pdfpagelabels=true%
   ]{hyperref}}   
\makeatother
%
\usepackage[]{bookmark}
%
\definecolor{pdfanchorcolor}{named}{black}
\definecolor{pdfmenucolor}{named}{red}
\definecolor{pdfruncolor}{named}{cyan}
\definecolor{pdfurlcolor}{rgb}{0,0,0.6}
\definecolor{pdffilecolor}{rgb}{0.7,0,0}
\definecolor{pdflinkcolor}{rgb}{0,0,0.6}
\definecolor{pdfcitecolor}{rgb}{0,0,0.6}
%
\hypersetup{
   ,draft=false, % all hypertext options are turned off
   ,final=true   % all hypertext options are turned on
   ,debug=false  % extra diagnostic messages are printed in the log file
   ,hypertexnames=true % use guessable names for links
   ,naturalnames=false % use LATEX-computed names for links
   ,setpagesize=true   % sets page size by special driver commands
   ,raiselinks=true    % forces commands to reflect the real height of the link 
   ,breaklinks=true    % Allows link text to break across lines
   ,pageanchor=true    % Determines whether every page is given an implicit
   ,plainpages=false   % Forces page anchors to be named by the arabic
   ,linktocpage=true   % make page number, not text, be link on TOC, LOF and LOT
   ,colorlinks=true    % Colors the text of links and anchors.
   ,linkcolor  =pdflinkcolor   % Color for normal internal links.
   ,anchorcolor=pdfanchorcolor % Color for anchor text.
   ,citecolor  =pdfcitecolor   % Color for bibliographical citations in text.
   ,filecolor  =pdffilecolor   % Color for URLs which open local files.
   ,menucolor  =pdfmenucolor   % Color for Acrobat menu items.
   ,runcolor   =pdfruncolor    % Color for run links (launch annotations).
   ,urlcolor   =pdfurlcolor    % color magenta Color for linked URLs.
   ,bookmarksopen=true     % If Acrobat bookmarks are requested, show them
   ,bookmarksopenlevel=2   % level (\maxdimen) to which bookmarks are open
   ,bookmarksnumbered=true %
   ,bookmarkstype=toc      %
   ,pdfpagemode=UseOutlines %
   ,pdfstartpage=1         % Determines on which page the PDF file is opened.
   ,pdfstartview=FitH      % Set the startup page view
   ,pdfremotestartview=Fit % Set the startup page view of remote PDF files
   ,pdfcenterwindow=false  %
   ,pdffitwindow=false     % resize document window to fit document size
   ,pdfnewwindow=false     % make links that open another PDF file 
   ,pdfdisplaydoctitle=true  % display document title instead of file name 
} % end: hypersetup
%
%%% color setup
% table colors 
\colorlet{tablebodycolor}{white!100}
\colorlet{tablerowcolor}{gray!10}
\colorlet{tableheadcolor}{gray!25}
%%% Set document layout / variables
\setlength{\parindent}{0pt}
\setlength{\parskip}{0.5\baselineskip}
\setcounter{secnumdepth}{2}
\setcounter{tocdepth}{2}
%%% doc commands
\newcommand{\package}[1]{\texttt{#1}}
\newcommand{\option}[1]{\texttt{#1}}
\newcommand{\parameter}[1]{\texttt{#1}}
\renewcommand\arg[1]{\meta{\normalfont\slshape#1}}
% 
\newcommand{\latex}{\LaTeX}
\newcommand{\Default}[1]{\par Default: \texttt{#1} \par}
\newcommand{\Example}[1]{\par Example: \texttt{#1} \par}

\newenvironment{Optionlist}{%
\begin{flushright}%
%  Style  changes
\small\renewcommand{\arraystretch}{1.4}%
%  tabu
\begin{tabu} to 1.0\linewidth  {>{\ttfamily}l<{\normalfont}X[1,l]}%
}{%
\end{tabu}%
\end{flushright}%
}


\begin{document}
  \DocInput{templatetools.dtx}
  \PrintChanges
  \PrintIndex
\end{document}
%</driver>
% \fi
%
% \CheckSum{272}
%
% \CharacterTable
%  {Upper-case    \A\B\C\D\E\F\G\H\I\J\K\L\M\N\O\P\Q\R\S\T\U\V\W\X\Y\Z
%   Lower-case    \a\b\c\d\e\f\g\h\i\j\k\l\m\n\o\p\q\r\s\t\u\v\w\x\y\z
%   Digits        \0\1\2\3\4\5\6\7\8\9
%   Exclamation   \!     Double quote  \"     Hash (number) \#
%   Dollar        \$     Percent       \%     Ampersand     \&
%   Acute accent  \'     Left paren    \(     Right paren   \)
%   Asterisk      \*     Plus          \+     Comma         \,
%   Minus         \-     Point         \.     Solidus       \/
%   Colon         \:     Semicolon     \;     Less than     \<
%   Equals        \=     Greater than  \>     Question mark \?
%   Commercial at \@     Left bracket  \[     Backslash     \\
%   Right bracket \]     Circumflex    \^     Underscore    \_
%   Grave accent  \`     Left brace    \{     Vertical bar  \|
%   Right brace   \}     Tilde         \~}
%
%
% \changes{v0.1}{<*date*>}{Converted to DTX file}
%
% \DoNotIndex{\newcommand,\newenvironment}
%
% \providecommand*{\url}{\texttt}
% \GetFileInfo{templatetools.dtx}
% \title{The \textsf{templatetools} package}
% \author{Matthias Pospiech \\ \url{matthias@pospiech.eu}}
% \date{\fileversion~from \filedate}
%
% \maketitle
% \begin{abstract}\noindent
% Collection of tools, which are helpful for the creation of a \LaTeX{}
% template if conditional paths for code execution are required. 
% \end{abstract}
%
% \section{Usage}
%
% \subsection{Commands}
% The following commands check if a command sequence is defined or not. 
%
% \DescribeMacro{\IfDefined} \marg{command}\marg{code defined} \AfterLastParam
% Executes the code if the command is defined. 
%
%
% \DescribeMacro{\IfUndefined} \marg{command}\marg{code undefined}
% \AfterLastParam 
% Executes the code if the command is not defined.
%
% \DescribeMacro{\IfElseDefined} \marg{command}\marg{defined}\marg{undefined}
% \AfterLastParam
% Executes either the code in the \emph{defined} bracket if the command is
% defined or in the \emph{undefined} bracket if the code is undefined.
%
% \DescribeMacro{\IfElseUndefined} \marg{command}\marg{undefined}\marg{defined}
% \AfterLastParam
% As \cs{IfElseDefined}, but with switched brackets for defined and undefined.
%
%
% \subsubsection{Example}
% The \cs{usepackage} code is only executed if the required \cs{upmu} command
% is defined.
% \iffalse
%<*example>
% \fi
\begin{lstlisting}[style=demostyle]
% Requires: Command \upmu
\IfDefined{upmu}{\usepackage[upmu]{gensymb}}
\end{lstlisting}
% \iffalse
%</example>
% \fi
%
%
% \DescribeMacro{\IfMultDefined} 
% \marg{list of commands}\marg{defined}\marg{undefined} \AfterLastParam
% Checks a comma separated list of commands before it executes the defined
% code path if all commands were defined or the undefined code path else.
%
% \subsubsection{Example}
% \iffalse
%<*example>
% \fi
\begin{lstlisting}[style=demostyle]
% Requires: Command longtable and rowcolors
\IfMultDefined{longtable,rowcolors}
  {... longtable with rowcolors ...}
  {Error: Neither longtable nor rowcolors are defined}
\end{lstlisting}
% \iffalse
%</example>
% \fi
%
% \subsection{Draft mode}
%
% The following commands check if draft mode is active or not.
%
% \DescribeMacro{\IfDraft}\marg{draft mode active}
% \AfterLastParam
%
% \DescribeMacro{\IfNotDraft}\marg{draft mode disabled}
% \AfterLastParam
%
% \DescribeMacro{\IfNotDraftElse}\marg{draft mode disabled}\marg{draft mode active}
% \AfterLastParam
%
% \subsubsection{Example}
% The \package{bookmark} is not loaded in draft mode:
% \iffalse
%<*example>
% \fi
\begin{lstlisting}[style=demostyle]
\IfNotDraft{\usepackage[]{bookmark}}
\end{lstlisting}
% \iffalse
%</example>
% \fi
%
%
% \subsection{Packages}
%
% These commands check if a package was loaded or not. 
% This can be achieved in different ways with commands from other packages.
% The key point of these commands here is that they work not only in 
% the preamble and include no @-char. 
% 
% \DescribeMacro{\IfPackageLoaded}\marg{package}\marg{is loaded}
% \AfterLastParam
%
% \DescribeMacro{\IfPackageNotLoaded}\marg{package}\marg{is not loaded}
% \AfterLastParam
%
% \DescribeMacro{\IfPackagesLoaded}\marg{list of packages}\marg{all are loaded}
% \AfterLastParam
%
% \DescribeMacro{\IfPackagesNotLoaded}\marg{list of packages}\marg{none is loaded}
% \AfterLastParam
%
% \subsubsection{Example}
%
% \iffalse
%<*example>
% \fi
\begin{lstlisting}[style=demostyle]
% Load epstopdf only if graphicx was loaded
\IfPackageLoaded{graphicx}{%
  \usepackage{epstopdf}
}
% Do not load subcaption if subfig was loaded (incompatible)
\IfPackageNotLoaded{subfig}{
  \usepackage{subcaption}[2011/08/17]
}
\end{lstlisting}
% \iffalse
%</example>
% \fi
%
%
% \subsection{Package Loading order}
% In \latex{} documents it is quite often essential
% to load packages in the right order to ensure that everything works.
% However this makes it impossible to group similar packages together.
%
% The following commands allow to execute code after or before a 
% specified package and thus also allows to load packages in a specified order
% using \cs{usepackage} commands.
%
% If the reference package was not loaded in the preamble
% the code will nevertheless be executed before \texttt{\cs{begin}\{document\}}
%
% 
% \DescribeMacro{\ExecuteAfterPackage}
% \marg{after this package}\marg{execute this code}\AfterLastParam
%
% \DescribeMacro{\ExecuteBeforePackage}
% \marg{before this package}\marg{execute this code}\AfterLastParam
%
% \subsubsection{Example}
% \package{cleveref} package must be loaded after package \package{hyperref}.
%
% \iffalse
%<*example>
% \fi
\begin{lstlisting}[style=demostyle]
% loading: must be loaded after hyperref and after varioref
\ExecuteAfterPackage{hyperref}{
% caption and cleveref incompatible in Versions before 2011/12/24
  \usepackage{cleveref}[2011/12/24]
}
\end{lstlisting}
% \iffalse
%</example>
% \fi
%
% \subsection{Tikz Library}
% Checks if a tikz library was loaded.
%
% \DescribeMacro{\IfTikzLibraryLoaded}\marg{library}\marg{if loaded}
%
% 
% \subsubsection{Example}
% Executes the code only of the tikz library was loaded.
%
% \iffalse
%<*example>
% \fi
\begin{lstlisting}[style=demostyle]
\IfTikzLibraryLoaded{lindenmayersystems}{%
% code origin: pgf/tikz manual
\begin{tikzpicture}
\pgfdeclarelindenmayersystem{Koch curve}{
   \rule{F -> F-F++F-F}
}
\shadedraw [top color=white, bottom color=blue!50, draw=blue!50!black]
           [l-system={Koch curve, step=2pt, angle=60, axiom=F++F++F, order=3}]
           lindenmayer system -- cycle;
\end{tikzpicture}
}%
\end{lstlisting}
% \iffalse
%</example>
% \fi
%
% \subsection{Column types}
%
% \LaTeX{} provides no tool to check for the existence of a column type. 
% This is provided by the following commands:
%
% \DescribeMacro{\IfColumntypeDefined}\marg{columntype character}\marg{is 
% defined}\marg{is undefined} \AfterLastParam
%
% ^^A which is also available as a command checking a comma separated list for 
% ^^A column type characters
%
% \DescribeMacro{\IfColumntypesDefined}
% \marg{columntype character list}\marg{is defined}\marg{is undefined}
% \AfterLastParam
%
% \subsubsection{Example}
% Executes the code only of the X column type is defined and the tabularx package was 
% loaded by checking that \cs{tabularx} is defined.
%
% \iffalse
%<*example>
% \fi
\begin{lstlisting}[style=demostyle]
\IfColumntypeDefined{X}{%
\IfDefined{tabularx}{%
%
\begin{tabularx}{0.9\textwidth}{llXX}
\hline
l & l & X & X \\\hline
%
left column & left column & 
text which is considerably longer than the width of the column & 
text which is considerably longer than the width of the column \\
\hline
\end{tabularx}
%
}}%
\end{lstlisting}
% \iffalse
%</example>
% \fi

% \subsection{Color definitions}
%
% Color definitions are saved in \latex{} as names. The following commands
% provide a convenient way to check the existence of these color definitions.
% 
% \DescribeMacro{\IfColorDefined} 
% \marg{color name}\marg{is defined}\marg{is undefined} \AfterLastParam
%
% ^^A Also available as a command checking for multiple color definitions with a 
% ^^A comma separated list: 
%
% \DescribeMacro{\IfColorsDefined}
% \marg{list of color name}\marg{is defined}\marg{is undefined} \AfterLastParam
%
% ^^A \subsection{Math font version}
% ^^A \DescribeMacro{\IfMathVersionDefined}
%
% \subsection{Template Definitions}
% The following commands in principle define only macros, but in contrast to 
% normal methods these are saved using two keys named \emph{group} 
% and \emph{property}. With a matching command for the execution this
% allows to generate macros in an object like naming structure, which
% can be used to toggle settings. 
%
% \DescribeMacro{\SetTemplateDefinition} \marg{Group}\marg{Property}\marg{Code}
% \AfterLastParam
% Defines a collection of commands (a macro) with a \emph{group} and 
% \emph{property}.
%
% \DescribeMacro{\UseDefinition} \marg{Group}\marg{Property} \AfterLastParam
% Execute macro save with the \emph{group} and \emph{property}.
%
% \subsubsection{Example}
% The following code allows to switch the colors anywhere in the document:
%
% \iffalse
%<*example>
% \fi
\begin{lstlisting}[style=demostyle]
\SetTemplateDefinition{Target}{Web}{%
  \definecolor{pdfurlcolor}{rgb}{0,0,0.6}
}%
\SetTemplateDefinition{Target}{Print}{%
  \definecolor{pdfurlcolor}{rgb}{0,0,0}
}%
% Apply colors for web
\UseDefinition{Target}{Web}
\end{lstlisting}
% \iffalse
%</example>
% \fi
%
% \StopEventually{}
% \clearpage
% \section{Implementation}
%
% \iffalse
%<*templatetools.sty>
% \fi
%
%
%    \begin{macrocode}
\NeedsTeXFormat{LaTeX2e}[1994/12/01]
\ProvidesPackage{templatetools}
          [<*date*> v0.1 Collection of conditional commands useful inside templates]
%    \end{macrocode}
%    \begin{macrocode}
%%% --- Necessary Packages
%%% ------------------------------------------------------------------
\RequirePackage{ifpdf}
\RequirePackage{etoolbox}
\RequirePackage{ltxcmds}
\RequirePackage{array} % for column types
\RequirePackage{ifdraft} % check draft
\RequirePackage{scrlfile}
%    \end{macrocode}
%
% \subsection{Command sequences}
%
% \iffalse
%%% --- Check if command is defined
%%% ------------------------------------------------------------------
% \fi
%
%
% \begin{macro}{\IfDefined}
% Wrapper to \cs{ifcsdef} with only true path.
%    \begin{macrocode}
\newcommand{\IfDefined}[2]{\ifcsdef{#1}{#2}{}}%
%    \end{macrocode}
% \end{macro}
%
%
% \begin{macro}{\IfUndefined}
% Wrapper to \cs{ifcsdef} with only false path.
%    \begin{macrocode}
\newcommand{\IfUndefined}[2]{\ifcsdef{#1}{}{#2}}%
%    \end{macrocode}
% \end{macro}
%
%
% \begin{macro}{\IfElseDefined}
% Wrapper to \cs{ifcsdef} with true and false path.
%    \begin{macrocode}
\newcommand{\IfElseDefined}[3]{\ifcsdef{#1}{#2}{#3}}%
%    \end{macrocode}
% \end{macro}
%
%
% \begin{macro}{\IfElseUndefined}
% Wrapper to \cs{ifcsdef} with true and false path in reverse order.
%    \begin{macrocode}
\newcommand{\IfElseUndefined}[3]{\ifcsdef{#1}{#3}{#2}}%
%    \end{macrocode}
% \end{macro}
%
%
% \begin{macro}{\IfMultDefined}
% Checks if more than one command is defined
%    \begin{macrocode}
\newcommand{\IfMultDefined}[1]{%
  \@tempswatrue
  \def\do##1{%
%% define \@tempa with trimmed index element.
    \edef\@tempa{\zap@space##1 \@empty}%
%% check if package of current index is loaded
    \ifcsdef{\@tempa}{}{\@tempswafalse}%
  }%
%% Process csv list with command \do (etoolbox)
  \docsvlist{#1}%
%% makes sure that the conditional works with one or two (if, else) parameters.
  \if@tempswa\expandafter\@firstoftwo\else\expandafter\@secondoftwo\fi%
}
%    \end{macrocode}
% \end{macro}
% 
% Thank to \person{egreg,Andrey Vihrov, Martin Scharrer} on 
%\href{http://tex.stackexchange.com/questions/35282/how-to-implement-a-command-that-checks-for-loaded-packages}{tex.stackexchange.com}
% for the help to implement a command that checks a comma separated list. 
%
% \subsection{Draft mode}
%
% \iffalse
% ------------------------------------------------------------------
% Check for 'draft' mode - commands.
% ------------------------------------------------------------------
% \fi
%
%
% \begin{macro}{\IfDraft}
% Tests if \cs{@draft} is undefined and executed false path in case draft string 
% is defined.
%    \begin{macrocode}
\newcommand{\IfDraft}[1]{\ifx\@draft\@undefined \else #1 \fi}
%    \end{macrocode}
% \end{macro}

%
% \begin{macro}{\IfNotDraft}
% Similar to \cs{IfDraft} but executes only path for draft mode undefined.
%    \begin{macrocode}
\newcommand{\IfNotDraft}[1]{\ifx\@draft\@undefined #1 \fi}
%    \end{macrocode}
% \end{macro}
%
%
% \begin{macro}{\IfNotDraftElse}
% Similar to \cs{IfDraft} but executes true and fals path.
%    \begin{macrocode}
\newcommand{\IfNotDraftElse}[2]{\ifx\@draft\@undefined #1 \else #2 \fi}
%    \end{macrocode}
% \end{macro}
%
%
%
% \subsection{Packages}
%
% \iffalse
% ------------------------------------------------------------------
% Check load status of packages
% ------------------------------------------------------------------
% \fi
%
% If a package is loaded can be checked in many ways, but here the 
% \cs{ltx@ifpackageloaded} is used because it can be executed anywhere
% in the document. 
%
% \begin{macro}{\IfPackageLoaded}
% Wrapper to \cs{ltx@ifpackageloaded} with only true path.
%    \begin{macrocode}
\newcommand{\IfPackageLoaded}[2]{\ltx@ifpackageloaded{#1}{#2}{}}
%    \end{macrocode}
% \end{macro}
%
%
% \begin{macro}{\IfPackageNotLoaded}
% Wrapper to \cs{ltx@ifpackageloaded} with only false path.
%    \begin{macrocode}
\newcommand{\IfPackageNotLoaded}[2]{\ltx@ifpackageloaded{#1}{}{#2}}
%    \end{macrocode}
% \end{macro}
%
% \begin{macro}{\IfElsePackageLoaded\}
% Wrapper to \cs{ltx@ifpackageloaded}
%    \begin{macrocode}
\let\IfElsePackageLoaded\ltx@ifpackageloaded
%    \end{macrocode}
% \end{macro}
%
% \begin{macro}{\IfPackagesLoaded}
% Checks a list of packages
%    \begin{macrocode}
\newcommand{\IfPackagesLoaded}[1]{%
  \@tempswatrue
  \def\do##1{%
    \edef\@tempa{\zap@space##1 \@empty}%
    \ltx@ifpackageloaded{\@tempa}{}{\@tempswafalse}%
  }%
  \docsvlist{#1}%
  \if@tempswa\expandafter\@firstoftwo\else\expandafter\@secondoftwo\fi%
}
%    \end{macrocode}
% \end{macro}
%
%
% \begin{macro}{\IfPackagesNotLoaded}
% Invers check if all packages in a list are not loaded
%    \begin{macrocode}
\newcommand{\IfPackagesNotLoaded}[1]{%
  \@tempswatrue
  \def\do##1{%
    \edef\@tempa{\zap@space##1 \@empty}%
    \ltx@ifpackageloaded{\@tempa}{\@tempswafalse}{}%
  }%
  \docsvlist{#1}%
  \if@tempswa\expandafter\@firstoftwo\else\expandafter\@secondoftwo\fi%
}
%    \end{macrocode}
% \end{macro}
%
%
% \begin{macro}{\ExecuteAfterPackage}
% Executes the code after the reference package has been loaded (using
% \cs{AfterAtEndOfPackage}) or finally at the end of the preamble
% if the reference package was not loaded until then.
%
%    \begin{macrocode}
\newcommand{\ExecuteAfterPackage}[2]{%
%% #1: after this package
%% #2: code to execute
  \AfterAtEndOfPackage{#1}{
    #2%
  }
  \AtEndPreamble{%
    \IfPackageNotLoaded{#1}{%
     #2%
    }
  }
} 
%    \end{macrocode}
% \end{macro}
%
%
% \begin{macro}{\ExecuteBeforePackage}
% Executes the code directly before the reference package is loaded (using
% \cs{BeforePackage}) or finally at the end of the preamble
% if the reference package was not loaded until then.
%
%    \begin{macrocode}
\newcommand{\ExecuteBeforePackage}[2]{%
%% #1: before this package
%% #2: code to execute
  \BeforePackage{#1}{
    #2%
  }
  \AtEndPreamble{%
    \IfPackageNotLoaded{#1}{%
     #2%
    }
  }
} 
%    \end{macrocode}
% \end{macro}
%
% \subsection{Tikz library}
%
% \iffalse
% ------------------------------------------------------------------
% The command sequence that is created for each tikz library when it is loaded is 
% used to check if it is loaded.
% \fi
%
%
% \begin{macro}{\IfTikzLibraryLoaded}
% Checks if the tikz library is loaded
%    \begin{macrocode}
\def\IfTikzLibraryLoaded#1{%
  \ifcsname tikz@library@#1@loaded\endcsname
    \expandafter\@firstoftwo
  \else
    \expandafter\@secondoftwo
  \fi
}
%    \end{macrocode}
% \end{macro}
%
% Thanks to \person{egreg} and \person{Marco Daniel} on \href{http://tex.stackexchange.com/questions/48453/how-to-test-if-tikz-library-is-loaded}{tex.stackexchange.com} for their help with this command.
%
% \subsection{Column types in tables}
%
% \iffalse
% ------------------------------------------------------------------
% table columns
% ------------------------------------------------------------------
% \fi
%
% The code in this section was inspired by the discussion with \person{egreg} on \href{http://tex.stackexchange.com/questions/35527/how-to-check-if-a-column-type-is-defined}{tex.stackexchange.com} on the detection of column definitions.
%
% Creates a list of predefined columntypes
%    \begin{macrocode}
\expandafter\let\csname columntype@l\endcsname\@empty
\expandafter\let\csname columntype@c\endcsname\@empty
\expandafter\let\csname columntype@r\endcsname\@empty
\expandafter\let\csname columntype@p\endcsname\@empty
\expandafter\let\csname columntype@m\endcsname\@empty
\expandafter\let\csname columntype@b\endcsname\@empty
\expandafter\let\csname columntype@@\endcsname\@empty
\expandafter\let\csname columntype@!\endcsname\@empty
\expandafter\let\csname columntype@|\endcsname\@empty
\expandafter\let\csname columntype@<\endcsname\@empty
\expandafter\let\csname columntype@>\endcsname\@empty
\expandafter\let\csname columntype@=\endcsname\@empty
%    \end{macrocode}
%
% \begin{macro}{\CheckIfColumntypeDefined}
% Creates a bool variable that saves the status of the column type.
%    \begin{macrocode}
\newcommand\CheckIfColumntypeDefined[1]{%
%% create the bool variable for column type
  \providebool{tpl@coltype@#1}
%% check if new column type of this name was created
  \ifcsdef{NC@find@\string#1}%
    {\setbool{tpl@coltype@#1}{true}}%
%% if not check if it is a predefined column type
    {\ifcsdef{columntype@\string#1}
      {\setbool{tpl@coltype@#1}{true}}%
      {\setbool{tpl@coltype@#1}{false}}%
    }%
}
%    \end{macrocode}
% \end{macro}
%
%
% \begin{macro}{\isColumntypeDefined}
% Returns the bool variable which can be interpreted by \cs{ifboolexpr}.
% This should only be used internally and fails for nonexistent bool variables.
%    \begin{macrocode}
\newcommand\isColumntypeDefined[1]{tpl@coltype@#1}
%    \end{macrocode}
% \end{macro}
%
%
% \begin{macro}{\IfColumntypeDefined}
% Executes \cs{CheckIfColumntypeDefined} and uses the resulting bool
% variable with \cs{isColumntypeDefined} in a conditional sequence with 
% \cs{ifboolexpr}.
%    \begin{macrocode}
\newcommand\IfColumntypeDefined[3]{%
%% Execute check which create bool variable
  \CheckIfColumntypeDefined{#1}
%% use bool variable for if sequence
  \ifboolexpr{ bool{\isColumntypeDefined{#1}} }{#2}{#3}%
}
%    \end{macrocode}
% \end{macro}
%
%
% \begin{macro}{\IfColumntypesDefined}
% Checks a comma separated list instead of a single string.
%    \begin{macrocode}
\newcommand{\IfColumntypesDefined}[1]{%
  \@tempswatrue
  \def\do##1{%
    \edef\@tempa{\zap@space##1 \@empty}%
%% check if column is defined.
%%   Here with \expandafter because of the \string definition 
%%   in \CheckIfColumntypeDefined. 
    \expandafter\IfColumntypeDefined
        \expandafter{\@tempa}{}{\@tempswafalse}
  }%
  \docsvlist{#1}%
  \if@tempswa\expandafter\@firstoftwo\else\expandafter\@secondoftwo\fi%
}
%    \end{macrocode}
% \end{macro}
%
%
% \subsection{Color definitions}
%
% \iffalse
% ------------------------------------------------------------------
% test color definitions
% ------------------------------------------------------------------
% \fi
%
%
% \begin{macro}{\IfColorDefined}
% Tests if a color is defined
%    \begin{macrocode}
\newcommand{\IfColorDefined}[3]{%
  \ifcsdef{\string\color @#1}
    {#2} % color string is defined
    {#3}} % color string is not defined
%    \end{macrocode}
% \end{macro}
%
%
% \begin{macro}{\IfColorsDefined}
% Does the same for a list (comma separated) of color names.
%    \begin{macrocode}
\newcommand{\IfColorsDefined}[1]{%
  \@tempswatrue
  \def\do##1{%
    \edef\@tempa{\zap@space##1 \@empty}%
    \expandafter\IfColorDefined
        \expandafter{\@tempa}{}{\@tempswafalse}
  }%
  \docsvlist{#1}%
  \if@tempswa\expandafter\@firstoftwo\else\expandafter\@secondoftwo\fi%
}
%    \end{macrocode}
% \end{macro}
%
% \subsection{Math font version}
%
% \iffalse
% ------------------------------------------------------------------
% test math versions
% ------------------------------------------------------------------
% \fi
%
% \begin{macro}{\IfMathVersionDefined}
% Checks if a mathversion font is defined.  
%    \begin{macrocode}
\newcommand{\IfMathVersionDefined}[3]{\ifcsdef{mv@#1}{#2}{#3}}
%    \end{macrocode}
% \end{macro}
%
% Thanks to \person{Werner} and \person{Ulrike Fischer} for pointing me to this
% name definition on 
% \href{http://tex.stackexchange.com/questions/40903/check-if-mathversion-font-is-defined}{tex.stackexchange.com}.
%
% Note, that this feature is untested.
%
% \subsection{Template definitions}
%
% \iffalse
% ------------------------------------------------------------------
% template definitions
% ------------------------------------------------------------------
% \fi
%
%
% \begin{macro}{\SetTemplateDefinition}
% Defines a macro with the \emph{group} and \emph{property} parameter names.
%    \begin{macrocode}
\newcommand\SetTemplateDefinition[3]{% 1: group, 2: property, 3: code
  \csdef{tpl@definition@#1@#2}{#3}
}
%    \end{macrocode}
% \end{macro}
%
%
% \begin{macro}{\UseDefinition}
% Executes the macro using \cs{csuse} if it is defined. Otherwise a error is 
% thrown.
%    \begin{macrocode}
\newcommand\UseDefinition[2]{%
  \ifcsdef{tpl@definition@#1@#2}
    {\csuse{tpl@definition@#1@#2}}
    {\PackageError{templatetools}%
      {Definition #1 -> #2 is unknown\MessageBreak}{}}%
}%
%    \end{macrocode}
% \end{macro}
%
%
% \iffalse
%</templatetools.sty>
% \fi
%
% \Finale
\endinput
