% \iffalse meta-comment
% !TeX spellcheck = en-US
%   
% File:      doctools.sty
% Version:   <*date*> v0.1
% Author:    Matthias Pospiech
% Email:     <matthias@pospiech.eu>
%
% Copyright (C) 2012 by Matthias Pospiech (matthias@pospiech.eu)
% ---------------------------------------------------------------------------
% This work may be distributed and/or modified under the
% conditions of the LaTeX Project Public License, either version 1.3
% of this license or (at your option) any later version.
% The latest version of this license is in
%   http://www.latex-project.org/lppl.txt
% and version 1.3 or later is part of all distributions of LaTeX
% version 2005/12/01 or later.
%
% This work has the LPPL maintenance status `maintained'.
%
% The Current Maintainer of this work is Matthias Pospiech.
%
% This work consists of the files doctools.dtx and doctools.ins
% and the derived filebase doctools.sty.
%
% \fi
% \iffalse
%<*driver>
\ProvidesFile{doctools.dtx}
%</driver>
%<package>\NeedsTeXFormat{LaTeX2e}[1999/12/01]
%<package>\ProvidesPackage{doctools}
%<*package>
   [<*date*> v0.1 commands and packages for documenting LaTeX Code]
%</package>
%
%<*driver>
\documentclass{ltxdoc}
\usepackage{hypdoc}
\RequirePackage[loadHyperref=true,createIndexEntries=false]{doctools} % 
\usepackage{lmodern}
%
%%% Configures doc.sty and its (outdated) index generation, which does
%   not work together with any modern package and has no hyperref support
%   and thus no links. Furthermore it conflicts with the index macros
%   of doctools.
%
\EnableCrossrefs  % (default) Every new macro name used within a macrocode or
                  % macrocode∗ environment will produce an index entry.
% \DisableCrossrefs % turn off this feature
%
% If an index is created is determined by the use of the following
% declarations in the driver file preamble; if neither is used, no index is 
% produced.
\PageIndex     % all index entries refer to their page number
% \CodelineIndex % index entries produced by \DescribeMacro and \DescribeEnv
               %  refer to page number but those produced by the macro
               %  environment refer to the code lines, 
               % which will be numbered automatically.
% \CodelineNumbered % no index is created, but the code lines are numbered

\RecordChanges
\listfiles
%
\begin{document}
  \DocInput{doctools.dtx}
  \PrintChanges
  \PrintIndex
\end{document}
%</driver>
% \fi
%
%
% \CharacterTable
%  {Upper-case    \A\B\C\D\E\F\G\H\I\J\K\L\M\N\O\P\Q\R\S\T\U\V\W\X\Y\Z
%   Lower-case    \a\b\c\d\e\f\g\h\i\j\k\l\m\n\o\p\q\r\s\t\u\v\w\x\y\z
%   Digits        \0\1\2\3\4\5\6\7\8\9
%   Exclamation   \!     Double quote  \"     Hash (number) \#
%   Dollar        \$     Percent       \%     Ampersand     \&
%   Acute accent  \'     Left paren    \(     Right paren   \)
%   Asterisk      \*     Plus          \+     Comma         \,
%   Minus         \-     Point         \.     Solidus       \/
%   Colon         \:     Semicolon     \;     Less than     \<
%   Equals        \=     Greater than  \>     Question mark \?
%   Commercial at \@     Left bracket  \[     Backslash     \\
%   Right bracket \]     Circumflex    \^     Underscore    \_
%   Grave accent  \`     Left brace    \{     Vertical bar  \|
%   Right brace   \}     Tilde         \~}
%
% \CheckSum{0}
%
% \changes{v0.1}{2012/12/01}{initial version (converted to dtx file)}
%
% \DoNotIndex{\newcommand,\newenvironment,\@bsphack,\@empty,\@esphack,\addtocounter,\arraystretch,\bs,\color,\colorlet,\else,\end,\expandafter,\fi,\hyperref,\ifcsdef,\ifstrempty,\ifx,\index,\label,\langle,\let,\linewidth,\lstinputlisting,\makeatletter,\makeatother,\NeedsTeXFormat,\newcounter,\noindent,\normalfont,\option,\par,\parindent,\parskip,\phantomsection,\PrintDescribeEnv,\PrintDescribeMacro,\PrintEnvName,\PrintMacroName,\ProcessKeyvalOptions,\providecommand,\ProvidesPackage,\rangle,\refstepcounter,\renewcommand,\RequirePackage,\setcounter,\setlength,\slshape,\small,\string,\strut,\textbackslash,\textsc,\texttt,\ttfamily,\urlstyle,\usepackage,\xspace}
%
% \providecommand*{\url}{\texttt}
% \GetFileInfo{doctools.dtx}
% \title{The \textsf{doctools} package}
% \author{<*author*> \\ \url{<*email*>}}
% \date{0.1~from <*date*>}
%
% \maketitle
% \tableofcontents
% ^^A ===================================================================
% \section{Documentation}
% This package is a collections of tools for the documentation of \latex code
% either in a normal document or within a dtx file.
% \StopEventually{}
%
% ^^A ~~~~~~~~~~~~~~~~~~~~~~~~~~~~~~~~~~~~~~~~~~~~~~~~~~~~~~~~~~~~~~~~~~~
% \subsection{Options}
% This package has two options 
% \begin{Optionlist}
%  \option{loadHyperref}   &  true/false (default: false)  \\
%  \option{createIndexEntries}   &  true/false (default: true)  \\
% \end{Optionlist}
% \package{doctools} is typically loaded before most other packages 
% and in the case of a dtx file possible the only package which 
% is necessary to load.
%
% The selection of these option should follow the following principles, which
% are mostly related to the fact that doc.sty does not support package 
% \package{hyperref} (prohibits display of index entries) and requires 
% the package \package{hypdoc} instead.
% 
% \subsubsection{Selection of options inside a dtx file}
% In a dtx file index entries are created by the package \package{doc.sty}
% which have a format the requires the use of the format file 
% \texttt{gind.ist} for makeindex:
%
% \iffalse
%<*example>
% \fi
\begin{lstlisting}[]
makeindex.exe -s gind.ist filename.idx
\end{lstlisting}
% \iffalse
%</example>
% \fi
%
% The index entries by \package{doc.sty} are created if \cs{PageIndex} or
% \cs{CodelineIndex} are defined except if \cs{CodelineNumbered} is defined.
% With the last definition no index is created, but the code lines are still
% numbered. In this case \package{hyperref} has no
% influence on the index creation since the index is never created: 
%
% \iffalse
%<*example>
% \fi
\begin{lstlisting}[style=lstDocStyleBase]
\CodelineNumbered
\usepackage[
  loadHyperref=true,%
  createIndexEntries=false% (has no influence)
]{doctools} % 
\end{lstlisting}
% \iffalse
%</example>
% \fi
%
% If in contrast the index entries of \package{doc.sty} are activated using \cs{PageIndex} or \cs{CodelineIndex}, then \package{hyperref} should not be loaded because it prohibits the display of the index entries created by \package{doc.sty}. Furthermore the index entries of \package{doctools} should be disabled because they use a different format and would be displayed wrong. 
%
% \iffalse
%<*example>
% \fi
\begin{lstlisting}[style=lstDocStyleBase]
\PageIndex
\usepackage[
  loadHyperref=false,% (would prohibit doc.sty index entries)
  createIndexEntries=false% (would be displayed wrong)
]{doctools} % 
\usepackage{hypdoc}
\end{lstlisting}
% \iffalse
%</example>
% \fi
% If the index entries of \package{doc.sty} are supposed to be displayed with hyperlinks the package \package{hypdoc} must be loaded instead\footnote{This issue is explained very well in this thread: \url{http://tex.stackexchange.com/questions/87670/disable-index-creation-in-dtx-file}}. If the option \option{loadHyperref} is used all styles are applied although \package{hyperref} is not loaded, because it is already loaded by the package \package{hypdoc}. Take care of the loading order: \package{hypdoc} must be loaded before \package{doctools}.
%
% \iffalse
%<*example>
% \fi
\begin{lstlisting}[style=lstDocStyleBase]
\PageIndex
\usepackage{hypdoc} % load before doctools !
\usepackage[
  loadHyperref=true,% apply hyperref styles
  createIndexEntries=false% (would be displayed wrong)
]{doctools} % 
\end{lstlisting}
% \iffalse
%</example>
% \fi
%
% However, if only the index entries of \package{doctools.sty} are supposed to be displayed in the index, this can be realized by loading \package{hyperref}. In this case the format file for \texttt{makeindex} must not be specified.
%
% \iffalse
%<*example>
% \fi
\begin{lstlisting}[style=lstDocStyleBase]
\PageIndex % creates index, but entries are not displayed
\usepackage[
  loadHyperref=true,% (prohibit doc.sty index entries)
  createIndexEntries=true% 
]{doctools} % 
\end{lstlisting}
% \iffalse
%</example>
% \fi
%
% \subsubsection{Selection of options in a normal \latex file}
% In a normal \latex file that does not load \package{doc.sty} the package options may be selected according to the needs of the user. If both index entries shall be created and \package{hyperref} can be safely loaded one could load the package with these options:
%
% \iffalse
%<*example>
% \fi
\begin{lstlisting}[style=lstDocStyleBase]
\usepackage[
  loadHyperref=true,% 
  createIndexEntries=true% 
]{doctools} % 
\end{lstlisting}
% \iffalse
%</example>
% \fi
%
% ^^A ~~~~~~~~~~~~~~~~~~~~~~~~~~~~~~~~~~~~~~~~~~~~~~~~~~~~~~~~~~~~~~~~~~~
% \subsection{Commands provided by doc.sty and ltxdoc}
% For the complete reference refer to \url{doc.pdf} and \url{ltxdoc.pdf}. Here % only the most relevant commands for the code in the documentation part are
% described.
%
% \DescribeMacro{\DescribeMacro} \marg{macro}  \AfterLastParam
% Prints out the macro name in the margin and
% adds the command to the index. It primary usage is to indicate the part of
% the documentation where the usage of this macro is described.
% 
% \DescribeMacro{\DescribeEnv} \marg{environment}  \AfterLastParam
% Is used analogues to \cs{DescribeMacro}.
%
% \DescribeMacro{\marg} \marg{argument}  \AfterLastParam
% Mandatory argument, printed as \marg{argument}
%
% \DescribeMacro{\oarg} \marg{argument}  \AfterLastParam
% Optional argument, printed as \oarg{argument}
%
% \DescribeMacro{\meta} Both \cs{marg} and \cs{oarg} use \cs{meta}
% to print out the argument as \meta{argument}.
%
% If \package{ltxdoc} is not used both \cs{marg} and \cs{oarg} are instead 
% defined by \package{doctools}.
%
% ^^A ~~~~~~~~~~~~~~~~~~~~~~~~~~~~~~~~~~~~~~~~~~~~~~~~~~~~~~~~~~~~~~~~~~~
% \subsection{Commands provided by doctools.sty}
% 
% \DescribeMacro{\bs} Shortcut for \cs{textbackslash}.
%
% \DescribeMacro{\command} \marg{cmd} \AfterLastParam
% Prints out the argument \cs{command}\arg{foo} as \cs{foo}.
%
% \DescribeMacro{\cs} \marg{cmd} \AfterLastParam
% Shortcut for \cs{command}. Also defined by \ltxclass{ltxdoc}.
%
% \DescribeMacro{\arg} \marg{cmd} \AfterLastParam
% Prints out an argument in curled brackets without the use of 
% angle brackets as in \cs{marg} or \cs{oarg}. 
% Thus prints \cs{arg}\arg{foo} as \arg{foo}.
%
% \DescribeMacro{\environment}  \marg{environment} \AfterLastParam
% Prints out an environment name as \environment{environment}.
%
% \DescribeMacro{\env} \marg{environment} \AfterLastParam
% Shortcut for \cs{environment}
%
% \DescribeMacro{\package} \marg{package} \AfterLastParam
% Prints out a package name as \package{package}.
%
% \DescribeMacro{\ltxclass} \marg{LaTeX documentclass} \AfterLastParam
% Prints out a class name as \ltxclass{class}.
%
% \DescribeMacro{\option} \marg{option} \AfterLastParam
% Prints out an option as \option{option}.
%
% \DescribeMacro{\parameter} \marg{parameter} \AfterLastParam
% Prints out an option as \parameter{parameter}.
%
% \DescribeMacro{\person} \marg{person} \AfterLastParam
%    Print out name of a person, for example for the acknowledge
%    of their help during the development of some code.
%    \Example{\normalfont This package was written by 
%    \person{Matthias Pospiech}.}
%
% \DescribeMacro{\AfterLastParam}
%    Used in conjuction with \cs{DescribeMacro} and the printout of arguments
%    using \cs{marg} or \cs{oarg}. \cs{AfterLastParam} is inserted at the end
%    to begin a new line.
%    \Example{\cs{DescribeMacro}\arg{foo}\cs{marg}\arg{argument}
%    \cs{AfterLastParam}}
%
% \DescribeMacro{\Default}
%    \cs{Default} is used to print out the default value of a command.
%
% \DescribeMacro{\Example}
%    \cs{Example} is used to print out an example for the usage of a command. %    See documentation of \cs{AfterLastParam} for an example.
%
% \DescribeEnv{Optionlist} Table to print out all possible options 
% of a command. See the following code for an example.
%
% \DescribeMacro{\latex}  New lower case command for printing out \latex.
%
%
% \iffalse
%<*example>
% \fi
\begin{latexcode}
\begin{Optionlist}
 top     &  placement is on top  \\
 bottom  &  placement is at the bottom\\
\end{Optionlist}
\end{latexcode}
% \iffalse
%</example>
% \fi
% Output:
% \begin{Optionlist}
%  top     &  placement is on top  \\
%  bottom  &  placement is at the bottom\\
% \end{Optionlist}
%
% ^^A ~~~~~~~~~~~~~~~~~~~~~~~~~~~~~~~~~~~~~~~~~~~~~~~~~~~~~~~~~~~~~~~~~~~
% \subsection{Printing of \latex code}
%
% This package provides two \package{listings} styles for the printing of 
% \latex code: \textsl{lstDemoStyleLaTeXCode} and
% \textsl{lstDocStyleLaTeXCode}. The first is meant for printing of code
% examples, the latter only to be used inside a dtx file for the documentation
% of the package code.
%
% For convenience these style are provided with the following environments.
%
% \DescribeEnv{latexcode} Listings environment for code examples. The output of this style is shown in the example below.
%
% \iffalse
%<*example>
% \fi
\begin{lstlisting}[style=lstDemoStyleLaTeXCode]
\begin{latexcode}
% comment
Example Code \ldots
\end{latexcode}
\end{lstlisting}
% \iffalse
%</example>
% \fi
%
% In the case of a dtx file this must be wrapped in special dtx code as shown % in the following code. 
%
% \iffalse
%<*example>
% \fi
\begin{lstlisting}[style=lstDemoStyleLaTeXCode]
% \iffalse
%<*example>
% \fi
\begin{latexcode}
Example Code \ldots
\end{latexcode}
% \iffalse
%</example>
% \fi
\end{lstlisting}
% \iffalse
%</example>
% \fi
%
% \DescribeEnv{macrocode} is in contrast an environment for the code in a dtx file, which is used for the creation of a latex package or class. In a dtx file therefore the special comment and indentation of the environment must be taken care of. The following example is displayed with this style.
% \iffalse
%<*example>
% \fi
\begin{lstlisting}[style=lstDocStyleLaTeXCode]
%    \begin{macrocode}
% definition of foobar
\newcommand{\foobar}{foo-bar}
%    \end{macrocode}
\end{lstlisting}
% \iffalse
%</example>
% \fi
%
% \DescribeMacro{\printCodeFromFile} 
% \oarg{first line} \marg{last line} \marg{file name} \AfterLastParam
% This command is very helpful to display parts of a file sequentially.
% If the first line is not given the last line value is incremented by one 
% and used as the first line.
% \Example{\cs{printCodeFromFile}[3]\arg{6}\arg{LaTeXTemplate.tex}}
%
% ^^A ~~~~~~~~~~~~~~~~~~~~~~~~~~~~~~~~~~~~~~~~~~~~~~~~~~~~~~~~~~~~~~~~~~~
% \subsection{Label Link system for files and their usage}
% The system explained in this section allows to adds a hyperlink to sections
% where a file is discussed and adds the usage page to the index. The code
% is very close to the label/ref system as shown in the following example:
%
% \iffalse
%<*example>
% \fi
\begin{latexcode}
\section{about links in pdf}    
The code is in \file{preamble/hyperref.tex}
\dots
\section{preamble/hyperref.tex}
\labelfile{preamble/hyperref.tex}
\end{latexcode}
% \iffalse
%</example>
% \fi
%
% \DescribeMacro{\labelfile} \marg{filepath/filename.tex} \AfterLastParam
% Creates a label that is used by \cs{file} for linking to the 
% occurrence of \cs{labelfile}. 
% Furthermore an index entry is created with a new sub-index for every path.
% In the case of \cs{labelfile}\arg{preamble/hyperref.tex} this would look
% like:
% \iffalse
%<*example>
% \fi
\begin{lstlisting}
Files
   preamble
      hyperref.tex
\end{lstlisting}
% \iffalse
%</example>
% \fi
%
% \DescribeMacro{\file} \marg{filepath/filename.tex} \AfterLastParam
% Prints out the filename in typewriter font and links to the usage if an
% equivalent \cs{labelfile} is given.
%
% ^^A =====================================================================
% \section{Implementation}
%
% \iffalse
%<*doctools.sty>
% \fi
%
%    \begin{macrocode}
\NeedsTeXFormat{LaTeX2e}
\ProvidesPackage{doctools}[2012/12/01 v0.1 commands and packages for documenting LaTeX Code]
%    \end{macrocode}
%
% ^^A ~~~~~~~~~~~~~~~~~~~~~~~~~~~~~~~~~~~~~~~~~~~~~~~~~~~~~~~~~~~~~~~~~~~
% \subsection{Define keys}
%
%    \begin{macrocode}
%%% === Define Keys ===================================
\RequirePackage{kvoptions-patch}
\RequirePackage{kvoptions}  % options
\RequirePackage{pdftexcmds} % string comparison
\SetupKeyvalOptions{family=doctools,prefix=doctools@}
%    \end{macrocode}
% Define default option for style key: \emph{stacked}
%    \begin{macrocode}
\DeclareBoolOption[false]{loadHyperref}
\DeclareBoolOption[false]{loadHypdoc}
\DeclareBoolOption[true]{createIndexEntries}
\ProcessKeyvalOptions{doctools}
%    \end{macrocode}
%
% ^^A ~~~~~~~~~~~~~~~~~~~~~~~~~~~~~~~~~~~~~~~~~~~~~~~~~~~~~~~~~~~~~~~~~~~
% \subsection{Preamble}
% \subsubsection{Packages}
% Most package are loaded at the beginning of the document to avoid clashes
% with other packages and make a correct loading order possible. Since the
% packages are only available at the end of the preamble, all commands
% defined by this package are also only available at the end of the preamble.
%    \begin{macrocode}
%%% ---- Packages ----
%%% Programming
\usepackage{etoolbox}
\usepackage{xstring}
\usepackage{kvsetkeys}
%%% Font packages
\usepackage{cmap} 
\usepackage[T1]{fontenc} % T1 Schrift Encoding
%\usepackage{lmodern} % Font not loaded, because this can lead
%                     % to incompatibilities with other math fonts
\usepackage{textcomp}
%%% listings (must be loaded before \AtBeginDocument)
\RequirePackage{listings}
%%% load all further packages and all commands 
%%% at the beginning of the document and thus
%%% after all other packages 
\PassOptionsToPackage{table}{xcolor}
\AtBeginDocument{%
%%
%%% Additional packages
\usepackage{xspace}
\@ifpackageloaded{xcolor}{}
 {\usepackage{xcolor}}
%%% listings
\colorlet{lstcolorStringLatex}{green!40!black!100}
\colorlet{lstcolorCommentLatex}{green!50!black!100}
\definecolor{lstcolorKeywordLatex}{rgb}{0,0.47,0.80}

% define useless command for checking the
% existens of this style
\newcommand{\lstStyleLaTeX}{\relax}
% define style
\lstdefinestyle{lstStyleLaTeX}{%
   ,style=lstStyleBase
%%% colors
   ,stringstyle=\color{lstcolorStringLatex}%
   ,keywordstyle=\color{lstcolorKeywordLatex}%
   ,commentstyle=\color{lstcolorCommentLatex}%
   ,% backgroundcolor=\color{codebackcolor}%
%%% Frames
   ,frame=single%
   ,%frameround=tttt%
   ,%framesep = 10pt%
   ,%framerule = 0pt%
   ,rulecolor = \color{black}%
%%% language
   ,language = [LaTeX]TeX%
%%% commands
   % LaTeX programming
   ,moretexcs={setlength,usepackage,newcommand,renewcommand,providecommand,RequirePackage,SelectInputMappings,ifpdftex,ifpdfoutput,AtBeginEnvironment,ProvidesPackage},
   % other commands
   ,moretexcs={maketitle,text,includegraphics,chapter,section,subsection,
     subsubsection,paragraph,textmu,enquote,blockquote,ding,mathds,ifcsdef,Bra,Ket,Braket,subcaption,lettrine,mdfsetup,captionof,listoffigures,listoftables,tableofcontents,appendix}
   % tables
   ,moretexcs={newcolumntype,rowfont,taburowcolors,rowcolor,rowcolors,bottomrule,
     toprule,midrule,}
   % hyperref
   ,moretexcs={hypersetup}
   % glossaries
   ,moretexcs={gls,printglossary,glsadd,newglossaryentry,newacronym}
   % Koma
   ,moretexcs={mainmatter,frontmatter,geometry,KOMAoptions,setkomafont,addtokomafont}
   % SI, unit
   ,moretexcs={si,SI,sisetup,unit,unitfrac,micro}
   % biblatex package
   ,moretexcs={newblock,ExecuteBibliographyOptions,addbibresource}
   % math packages
   ,moretexcs={operatorname,frac,sfrac,dfrac,DeclareMathOperator,mathcal,underset}
   % demo package
   ,moretexcs={democodefile,package,cs,command,env,DemoError,PrintDemo}  
   % tablestyles
   ,moretexcs={theadstart,tbody,tsubheadstart,tsubhead,tend}
   % code section package
   ,moretexcs={DefineTemplateSection,SetTemplateSection,BeginTemplateSection,
     EndTemplateSection}
   % template tools package
   ,moretexcs={IfDefined,IfUndefined,IfElseDefined,IfElseUndefined,IfMultDefined,IfNotDraft,IfNotDraftElse,IfDraft,IfPackageLoaded,IfElsePackageLoaded,IfPackageNotLoaded,IfPackagesLoaded,IfPackagesNotLoaded,ExecuteAfterPackage,ExecuteBeforePackage,IfTikzLibraryLoaded,IfColumntypeDefined,IfColumntypesDefined,IfColorDefined,IfColorsDefined,IfMathVersionDefined,SetTemplateDefinition,UseDefinition,IfFileExists,iflanguage}
   % tablestyles
   ,moretexcs={setuptablefontsize,tablefontsize,setuptablestyle,tablestyle,  setuptablecolor,tablecolor,disablealternatecolors,   tablealtcolored,tbegin,tbody,tend,thead, theadstart,tsubheadstart,tsubhead,theadrow,tsubheadrow,resettablestyle,theadend,tsubheadend,tableitemize,PreserveBackslash}
   % todonotes
   ,moretexcs={todo,missingfigure}
   % listings
   ,moretexcs={lstloadlanguages,lstdefinestyle,lstset}
   % index
   ,moretexcs={indexsetup}
   % glossaries
   ,moretexcs={newglossarystyle,glossarystyle,deftranslation,newglossary}
   % tikz
   ,moretexcs={usetikzlibrary}
   % color
   ,moretexcs={definecolor,colorlet}
   % caption
   ,moretexcs={captionsetup,DeclareCaptionStyle}
   % floatrow
   ,moretexcs={floatsetup}
   % doc.sty
   ,moretexcs={EnableCrossrefs,DisableCrossrefs,PageIndex,CodelineIndex,CodelineNumbered}   
}
% \ifcsdef{addmoretexcs}{%
% LaTeX programming
\addmoretexcs[LaTeX]{setlength,usepackage,newcommand,renewcommand,providecommand,RequirePackage,SelectInputMappings,ifpdftex,ifpdfoutput,AtBeginEnvironment,ProvidesPackage}
% other commands
\addmoretexcs[LaTeX]{maketitle,text,includegraphics,chapter,section,subsection,
subsubsection,paragraph,textmu,enquote,blockquote,ding,mathds,ifcsdef,Bra,Ket,Braket,subcaption,lettrine,mdfsetup,captionof,listoffigures,listoftables,tableofcontents,appendix,url}
% tables
\addmoretexcs[LaTeX]{newcolumntype,rowfont,taburowcolors,rowcolor,rowcolors,bottomrule,toprule,midrule}
% hyperref
\addmoretexcs[LaTeX]{hypersetup}
% glossaries
\addmoretexcs[LaTeX]{gls,printglossary,glsadd,newglossaryentry,newacronym}
% Koma
\addmoretexcs[LaTeX]{mainmatter,frontmatter,geometry,KOMAoptions,setkomafont,addtokomafont}
% SI, unit
\addmoretexcs[LaTeX]{si,SI,sisetup,unit,unitfrac,micro}
% biblatex package
\addmoretexcs[LaTeX]{newblock,ExecuteBibliographyOptions,addbibresource}
% math packages
\addmoretexcs[LaTeX]{operatorname,frac,sfrac,dfrac,DeclareMathOperator,mathcal,underset}
% demo package
\addmoretexcs[LaTeX]{democodefile,package,cs,command,env,DemoError,PrintDemo}  
% tablestyles
\addmoretexcs[LaTeX]{theadstart,tbody,tsubheadstart,tsubhead,tend}
% code section package
\addmoretexcs[LaTeX]{DefineCodeSection,SetCodeSection,BeginCodeSection,EndCodeSection}
% template tools package
\addmoretexcs[LaTeX]{IfDefined,IfUndefined,IfElseDefined,IfElseUndefined,IfMultDefined,IfNotDraft,IfNotDraftElse,IfDraft,IfPackageLoaded,IfElsePackageLoaded,IfPackageNotLoaded,IfPackagesLoaded,IfPackagesNotLoaded,ExecuteAfterPackage,ExecuteBeforePackage,IfTikzLibraryLoaded,IfColumntypeDefined,IfColumntypesDefined,IfColorDefined,IfColorsDefined,IfMathVersionDefined,SetTemplateDefinition,UseDefinition,IfFileExists,iflanguage}
% tablestyles
\addmoretexcs[LaTeX]{setuptablefontsize,tablefontsize,setuptablestyle,tablestyle,setuptablecolor,tablecolor,disablealternatecolors,tablealtcolored,tbegin,tbody,tend,thead,theadstart,tsubheadstart,tsubhead,theadrow,tsubheadrow,resettablestyle,theadend,tsubheadend,tableitemize,PreserveBackslash}
% todonotes
\addmoretexcs[LaTeX]{todo,missingfigure}
% listings
\addmoretexcs[LaTeX]{lstloadlanguages,lstdefinestyle,lstset}
% index
\addmoretexcs[LaTeX]{indexsetup}
% glossaries
\addmoretexcs[LaTeX]{newglossarystyle,glossarystyle,deftranslation,newglossary}
% tikz
\addmoretexcs[LaTeX]{usetikzlibrary}
% color
\addmoretexcs[LaTeX]{definecolor,colorlet}
% caption
\addmoretexcs[LaTeX]{captionsetup,DeclareCaptionStyle}
% floatrow
\addmoretexcs[LaTeX]{floatsetup}
% doc.sty
\addmoretexcs[LaTeX]{EnableCrossrefs,DisableCrossrefs,PageIndex,CodelineIndex,CodelineNumbered}
% refereces
\addmoretexcs[LaTeX]{cref,Cref,vref,eqnref,figref,tabref,secref,chapref}
%
}{} % end of \ifcsdef

\lstloadlanguages{[LaTeX]TeX}
%%% hyperref
\ifdoctools@loadHyperref
\makeatletter
\@ifpackageloaded{hypdoc}{}
   % load hyperref only if package 
   % hypdoc is not loaded, which
   % loads hyperref itself
   {\usepackage[
     ,backref=page%
     ,pagebackref=false%
     ,hyperindex=true%
     ,hyperfootnotes=false%
     ,bookmarks=true%
     ,pdfpagelabels=true%
   ]{hyperref}}   
\makeatother
%
\usepackage[]{bookmark}
%
\definecolor{pdfanchorcolor}{named}{black}
\definecolor{pdfmenucolor}{named}{red}
\definecolor{pdfruncolor}{named}{cyan}
\definecolor{pdfurlcolor}{rgb}{0,0,0.6}
\definecolor{pdffilecolor}{rgb}{0.7,0,0}
\definecolor{pdflinkcolor}{rgb}{0,0,0.6}
\definecolor{pdfcitecolor}{rgb}{0,0,0.6}
%
\hypersetup{
   ,draft=false, % all hypertext options are turned off
   ,final=true   % all hypertext options are turned on
   ,debug=false  % extra diagnostic messages are printed in the log file
   ,hypertexnames=true % use guessable names for links
   ,naturalnames=false % use LATEX-computed names for links
   ,setpagesize=true   % sets page size by special driver commands
   ,raiselinks=true    % forces commands to reflect the real height of the link 
   ,breaklinks=true    % Allows link text to break across lines
   ,pageanchor=true    % Determines whether every page is given an implicit
   ,plainpages=false   % Forces page anchors to be named by the arabic
   ,linktocpage=true   % make page number, not text, be link on TOC, LOF and LOT
   ,colorlinks=true    % Colors the text of links and anchors.
   ,linkcolor  =pdflinkcolor   % Color for normal internal links.
   ,anchorcolor=pdfanchorcolor % Color for anchor text.
   ,citecolor  =pdfcitecolor   % Color for bibliographical citations in text.
   ,filecolor  =pdffilecolor   % Color for URLs which open local files.
   ,menucolor  =pdfmenucolor   % Color for Acrobat menu items.
   ,runcolor   =pdfruncolor    % Color for run links (launch annotations).
   ,urlcolor   =pdfurlcolor    % color magenta Color for linked URLs.
   ,bookmarksopen=true     % If Acrobat bookmarks are requested, show them
   ,bookmarksopenlevel=2   % level (\maxdimen) to which bookmarks are open
   ,bookmarksnumbered=true %
   ,bookmarkstype=toc      %
   ,pdfpagemode=UseOutlines %
   ,pdfstartpage=1         % Determines on which page the PDF file is opened.
   ,pdfstartview=FitH      % Set the startup page view
   ,pdfremotestartview=Fit % Set the startup page view of remote PDF files
   ,pdfcenterwindow=false  %
   ,pdffitwindow=false     % resize document window to fit document size
   ,pdfnewwindow=false     % make links that open another PDF file 
   ,pdfdisplaydoctitle=true  % display document title instead of file name 
} % end: hypersetup
%
\fi
%    \end{macrocode}
%
% \subsubsection{color setup}
%
%    \begin{macrocode}
%%% color setup
% table colors 
\ifcsdef{colorlet}{%
\colorlet{tablebodycolor}{white!100}
\colorlet{tablerowcolor}{gray!10}
\colorlet{tableheadcolor}{gray!25}
}{}
%    \end{macrocode}
%
% \subsubsection{Document layout / variables}
%
%    \begin{macrocode}
%%% Set document layout / variables
\setlength{\parindent}{0pt}
\setlength{\parskip}{0.5\baselineskip}
\setcounter{secnumdepth}{2}
\setcounter{tocdepth}{2}
%    \end{macrocode}
%
% ^^A ~~~~~~~~~~~~~~~~~~~~~~~~~~~~~~~~~~~~~~~~~~~~~~~~~~~~~~~~~~~~~~~~~~~
% \subsection{Internal Variables}
%
%    \begin{macrocode}
%%% ---- Internal Variables ---- 
%% Font for Index Headings
\newcommand{\doctools@indexHeadFont}[1]{\textsc{#1}}
%    \end{macrocode}
%
% ^^A ~~~~~~~~~~~~~~~~~~~~~~~~~~~~~~~~~~~~~~~~~~~~~~~~~~~~~~~~~~~~~~~~~~~
% \subsection{Commands}
%
% \begin{macro}{\bs}
% Shortcut for \cs{textbackslash}.
%    \begin{macrocode}
%%% ---- Commands ---- 
%% \bs
\newcommand{\bs}{\textbackslash}
%    \end{macrocode}
% \end{macro}
%
% \begin{macro}{\command}
% Prints the argument with backslash in typewriter font. 
% This is meant to be used for \LaTeX{} commands.
% Additionally the command is added to the index.
%    \begin{macrocode}
%% \command
\newcommand{\command}[1]{%
\texttt{\textbackslash{}#1}\relax%
\ifdoctools@createIndexEntries%
  \index{\doctools@indexHeadFont{Command}!\textbackslash{}#1}%
\fi%
}%
%    \end{macrocode}
% \end{macro}
%
% \begin{macro}{\cs}
% Shortcut for \cs{command}
%    \begin{macrocode}
%% \cs (shortcut for \command)
%% \cs might be defined by ltxdoc, therefore it needs to be deleted
%% before it can be redefined.
\ifcsdef{cs}{\csundef{cs}}{}%
\let\cs\command%
%    \end{macrocode}
% \end{macro}
%
% \begin{macro}{\environment}
% Formats an environment in typewriter font and adds it to the index. 
%    \begin{macrocode}
%% \environment
\ifcsdef{environment}{\csundef{environment}}{}
\newcommand{\environment}[1]{%
\texttt{#1}%
\ifdoctools@createIndexEntries
  \index{\doctools@indexHeadFont{Environment}!#1}%
\fi
}%
%    \end{macrocode}
% \end{macro}
%
% \begin{macro}{\env}
% Shortcut for \cs{environment}
%    \begin{macrocode}
%% \env
\newcommand{\env}[1]{\environment{#1}}
%    \end{macrocode}
% \end{macro}
%
% \begin{macro}{\package}
% Formats an package in typewriter font and adds it to the index. 
%    \begin{macrocode}
%% \package
\newcommand{\package}[1]{%
\texttt{#1}%
\ifdoctools@createIndexEntries
\index{\doctools@indexHeadFont{Package}!#1}%
\fi
}%
%    \end{macrocode}
% \end{macro}
%
% \begin{macro}{\ltxclass}
% Formats a \latex class in typewriter font and adds it to the index. 
%    \begin{macrocode}
%% \ltxclass
\newcommand{\ltxclass}[1]{%
\texttt{#1}%
\ifdoctools@createIndexEntries
\index{\doctools@indexHeadFont{Class}!#1}%
\fi
}%
%    \end{macrocode}
% \end{macro}
%
% \begin{macro}{\marg}
% Formats a mandatory argument of a command. If \cs{meta} is defined this is reused.
%    \begin{macrocode}
%% \marg
\ifcsdef{marg}{}{% if not defined
\ifcsdef{meta}%
  {\newcommand\marg[1]{\texttt{\{}\meta{#1}\texttt{\}}}}%
  {\newcommand\marg[1]{
\texttt{\{}%
$\langle${\normalfont\slshape#1}$\rangle$
\texttt{\}}}}%
}%
%    \end{macrocode}
% \end{macro}
%
% \begin{macro}{\oarg}
% Formats a optional argument of a command. If \cs{meta} is defined this is reused.
%    \begin{macrocode}
%% \oarg
\ifcsdef{oarg}{}{% if not defined
\ifcsdef{meta}%
  {\newcommand\oarg[1]{\texttt{[}\meta{#1}\texttt{]}}}%
  {\newcommand\oarg[1]{
\texttt{[}%
$\langle${\normalfont\slshape#1}$\rangle$
\texttt{]}}}%
}%
%    \end{macrocode}
% \end{macro}
%
% \begin{macro}{\arg}
% Formats an argument of a command without extra angle brackets 
% in curled brackets in monospaced font.
%    \begin{macrocode}
%% \arg
\ifcsdef{arg}{%
  \csundef{arg}%
  \newcommand\arg[1]{\{\texttt{#1}\}}
}{}%
%    \end{macrocode}
% \end{macro}
%
% \begin{macro}{\option}
% Formats an option in typewriter font and adds it to the index. 
%    \begin{macrocode}
%% \option
\newcommand{\option}[1]{%
\texttt{#1}%
\ifdoctools@createIndexEntries
\index{option!#1}%
\fi
}%
%    \end{macrocode}
% \end{macro}
%
% \begin{macro}{\parameter}
% Formats a parameter in typewriter font. 
%    \begin{macrocode}
%% \parameter
\newcommand{\parameter}[1]{%
\texttt{#1}%
}%
%    \end{macrocode}
% \end{macro}
%
% \begin{macro}{\person}
%    Print out name of a person, for example for the acknowledge
%    of their help during the development of some code.
%    \begin{macrocode}
%% \person
\newcommand{\person}[1]{\textsc{#1}}
%    \end{macrocode}
% \end{macro}
%
% \begin{macro}{\AfterLastParam}
%    Used in conjuction with \cs{DescribeMacro} and the printout of arguments
%    using \cs{marg} or \cs{oarg}. \cs{AfterLastParam} is inserted at the end
%    to begin a new line.
%    \begin{macrocode}
%% \AfterLastParam
\newcommand{\AfterLastParam}{\par\noindent}
%    \end{macrocode}
% \end{macro}
%
% \begin{macro}{\Default}
%    \cs{Default} is used to print out the default value of a command.
%    \begin{macrocode}
%% \Default
\newcommand{\Default}[1]{\par Default: \texttt{#1} \par}
%    \end{macrocode}
% \end{macro}
%
% \begin{macro}{\Example}
%    \cs{Example} is used to print out an example for the usage of a command.
%    \begin{macrocode}
%% \Example
\newcommand{\Example}[1]{\par Example: \texttt{#1} \par}
%    \end{macrocode}
% \end{macro}
%
% ^^A ~~~~~~~~~~~~~~~~~~~~~~~~~~~~~~~~~~~~~~~~~~~~~~~~~~~~~~~~~~~~~~~~~~~
% \subsection{Option list}
%
% \begin{macro}{Optionlist}
%    Environment
%    \begin{macrocode}
\newenvironment{Optionlist}{%
\begin{flushleft}%
\vspace{-1\baselineskip}
%%  Style  changes
\small\renewcommand{\arraystretch}{1.4}%
%%  table
\begin{tabular} {>{\ttfamily}l<{\normalfont}p{0.7\textwidth}}%
}{% end of environment
\end{tabular}%
\end{flushleft}%
}%
%    \end{macrocode}
% \end{macro}
%
% ^^A ~~~~~~~~~~~~~~~~~~~~~~~~~~~~~~~~~~~~~~~~~~~~~~~~~~~~~~~~~~~~~~~~~~~
% \subsection{\LaTeX{} engine names}
%
% \begin{macro}{\latex}
%    New lower case command for printing out \LaTeX{}
%    \begin{macrocode}
%%% ---- engine names ----
%% \latex
% ensure that space is added after \latex
\newcommand{\latex}{\LaTeX\xspace}
%    \end{macrocode}
% \end{macro}
%
%
% ^^A ~~~~~~~~~~~~~~~~~~~~~~~~~~~~~~~~~~~~~~~~~~~~~~~~~~~~~~~~~~~~~~~~~~~
% \subsection{doc.sty modifications}
% The following commands from doc.sty are changed by
% introducing colors to the commands
%    \begin{macrocode}
%%% ---- doc.sty modifications ----
% define color for Macro and Environment names
\colorlet{doctools@ColorCodeNames}{blue!50!black}
%%% Overwrite font for \meta
\def\meta@font@select{\normalfont\slshape} % original: \itshape
%
\ifcsdef{PrintMacroName}
   {\def\PrintMacroName#1{\strut \MacroFont %
   \color{doctools@ColorCodeNames}\string #1\ }}{}
%
\newcounter{MacroName} % hyperref uses \theH<counter>
\providecommand*{\theHMacroName}{\theMacroName}
%
%% create label 
%   \renewcommand*{\theHMacroName}{#1}%
%   \ifcsdef{phantomsection}{\phantomsection}{}%
%   \@bsphack%
%   \refstepcounter{MacroName}%
%   \label{doc:desc:#1}%   
\ifcsdef{PrintDescribeMacro}
   {\def\PrintDescribeMacro#1{%
   \strut \MacroFont %
   \color{doctools@ColorCodeNames} \string #1\ }}{}
\ifcsdef{PrintDescribeEnv}
   {\def\PrintDescribeEnv#1{\strut \MacroFont %
   \color{doctools@ColorCodeNames} #1\ }}{}
\ifcsdef{PrintEnvName}
   {\def\PrintEnvName#1{\strut \MacroFont %
   \color{doctools@ColorCodeNames} #1\ }}{}
%
%%% disable the index preamble if index entries are generated by this package
\ifdoctools@createIndexEntries
\ifcsdef{index@prologue}
     {\def\index@prologue{\section*{Index}\markboth{Index}{Index}}}
     {}
\fi
%     
%    \end{macrocode}
% ^^A ~~~~~~~~~~~~~~~~~~~~~~~~~~~~~~~~~~~~~~~~~~~~~~~~~~~~~~~~~~~~~~~~~~~
% \subsection{Listings environments}
%
%    \begin{macrocode}
%%% ---- listings environments for LaTeX code
%%% Overwriting the environment macrocode for printing the code in a dtx file.
\ifcsdef{macrocode}{\csundef{macrocode}}{}
\lstnewenvironment{macrocode}{\lstset{style=lstDocStyleLaTeXCode}}{}
%    \end{macrocode}
%
%    \begin{macrocode}
%%% environment code examples.
\lstnewenvironment{latexcode}{\lstset{style=lstDemoStyleLaTeXCode}}{}
%    \end{macrocode}
%
% ^^A ~~~~~~~~~~~~~~~~~~~~~~~~~~~~~~~~~~~~~~~~~~~~~~~~~~~~~~~~~~~~~~~~~~~
% \subsection{Print LaTeX code from external file using listings}
%
%    \begin{macrocode} 
% -------------------------------------------------
% \printCodeFromFile
% -------------------------------------------------
\newcounter{lstFirstLine}
\newcounter{lstLastLine}
\setcounter{lstLastLine}{0}
\setcounter{lstFirstLine}{0}
%
\newcommand{\printCodeFromFile}[3][]{%
\ifstrempty{#1}{}{%
  \setcounter{lstFirstLine}{#1}%
}%
\setcounter{lstLastLine}{#2}%
%
\lstinputlisting[%
  firstnumber=\thelstFirstLine,%
  firstline=\thelstFirstLine,%
  lastline=\thelstLastLine,%
  nolol=true,
  style=lstDemoStyleLaTeXCode]%
  {#3}%
%
%% set counter to lastLine + 1
\setcounter{lstFirstLine}{\thelstLastLine}
\addtocounter{lstFirstLine}{1}
}
%    \end{macrocode}
%
% ^^A ~~~~~~~~~~~~~~~~~~~~~~~~~~~~~~~~~~~~~~~~~~~~~~~~~~~~~~~~~~~~~~~~~~~
% \subsection{References and links to files}
%
%    \begin{macrocode}
% -------------------------------------------------
% \file and \labelfile
% -------------------------------------------------
% Code copied from http://tex.stackexchange.com/
%  questions/65639/how-to-create-my-on-ref-label-system/
% with later modifications.
% Thanks to Heiko Oberdiek for providing this answer !
% -------------------------------------------------
%% Command for printing the filename
\ifcsdef{urlstyle}{}{\usepackage{url}}
\DeclareUrlCommand{\PrintFileName}{\urlstyle{tt}}
%%
\newcounter{file}
%% hyperref uses \theH<counter>
\providecommand*{\theHfile}{\thefile}
%% code from tex.stackexchange with the 
%% the help from Heiko Oberdiek.
\newcommand*{\labelfile}[1]{%
  %% convert all "/" to "/," (comma list) and save in \IndexFileA
  \StrSubstitute{#1}{/}{/,}[\IndexFileA]%
  %% define \IndexFileB as empty (used for the output string)
  \let\IndexFileB\@empty
  %% parse and convert \IndexFileA using \@AddFileEntry
  \expandafter
  \comma@parse@normalized\expandafter{\IndexFileA}\@AddFileEntry
  %% create label and print to index
  \@bsphack
  \renewcommand*{\theHfile}{#1}%
  \refstepcounter{file}%
  \ifcsdef{phantomsection}{\phantomsection}{}
  \label{file:#1}%
  \ifdoctools@createIndexEntries
  \index{\textsc{Files}!\IndexFileB}%
  \fi
  \@esphack  
}
%% add entries of comma list to the index.
\newcommand*{\@AddFileEntry}[1]{%
  \ifx\IndexFileB\@empty
    %% add first entry to index with lexEntry@{printEntry}
    \def\IndexFileB{#1@#1}%
  \else
    \expandafter\def\expandafter\IndexFileB\expandafter{%
    %% add following entries to index with 
    %% previousEntries!lexEntry@{printEntry}
\IndexFileB!%
#1@#1%
    }%
  \fi
}
%% Print out filename and create a link if the label exists.
\newcommand*{\file}[1]{%
  \ifcsdef{hyperref}%
    {\hyperref[file:#1]{\PrintFileName{#1}}}%
    {\PrintFileName{#1}}%
}% 
%%
} % end of \AtBeginDocument 

%    \end{macrocode}
%
%
%
% \iffalse
%</doctools.sty>
% \fi
% \clearpage
% \Finale
\endinput
