% \iffalse meta-comment
%  
% File:      templatedemo.sty
% Version:   2012/06/01 v0.1
% Author:    Matthias Pospiech
% Email:     <matthias@pospiech.eu>
%
% Copyright (C) 2012 by Matthias Pospiech (matthias@pospiech.eu)
% ---------------------------------------------------------------------------
% This work may be distributed and/or modified under the
% conditions of the LaTeX Project Public License, either version 1.3
% of this license or (at your option) any later version.
% The latest version of this license is in
%   http://www.latex-project.org/lppl.txt
% and version 1.3 or later is part of all distributions of LaTeX
% version 2005/12/01 or later.
%
% This work has the LPPL maintenance status `maintained'.
%
% The Current Maintainer of this work is Matthias Pospiech.
%
% This work consists of the files templatedemo.dtx and templatedemo.ins
% and the derived filebase templatedemo.sty.
%
% \fi
% \newcommand\packagetext{templatedemo}
% \newcommand\authortext{Matthias Pospiech}
% \newcommand\emailtext{matthias@pospiech.eu} 
% \newcommand\versiontext{v0.1} 
% \newcommand\datetext{2012/06/01} 
% \iffalse
%<*driver>
\ProvidesFile{templatedemo.dtx}
%</driver>
%<package>\NeedsTeXFormat{LaTeX2e}[1999/12/01]
%<package>\ProvidesPackage{templatedemo}
%<*package>
   [2012/05/20 v0.1 typeset code and resulting output]
%</package>
%
%<*driver>
\documentclass{ltxdoc}
\RequirePackage{templatedemo}
\EnableCrossrefs
\CodelineIndex
\RecordChanges

%%% Font packages
\usepackage{cmap} 
\usepackage[T1]{fontenc} % T1 Schrift Encoding
\usepackage{lmodern}
%%% Additional packages
\usepackage{soulutf8}
\usepackage[table]{xcolor}
\usepackage{enumitem}
\usepackage{tabu}
%%% hyperref
\makeatletter
\@ifpackageloaded{hypdoc}{}
   % load hyperref only if package 
   % hypdoc is not loaded, which
   % loads hyperref itself
   {\usepackage[
     ,backref=page%
     ,pagebackref=false%
     ,hyperindex=true%
     ,hyperfootnotes=false%
     ,bookmarks=true%
     ,pdfpagelabels=true%
   ]{hyperref}}   
\makeatother
%
\usepackage[]{bookmark}
%
\definecolor{pdfanchorcolor}{named}{black}
\definecolor{pdfmenucolor}{named}{red}
\definecolor{pdfruncolor}{named}{cyan}
\definecolor{pdfurlcolor}{rgb}{0,0,0.6}
\definecolor{pdffilecolor}{rgb}{0.7,0,0}
\definecolor{pdflinkcolor}{rgb}{0,0,0.6}
\definecolor{pdfcitecolor}{rgb}{0,0,0.6}
%
\hypersetup{
   ,draft=false, % all hypertext options are turned off
   ,final=true   % all hypertext options are turned on
   ,debug=false  % extra diagnostic messages are printed in the log file
   ,hypertexnames=true % use guessable names for links
   ,naturalnames=false % use LATEX-computed names for links
   ,setpagesize=true   % sets page size by special driver commands
   ,raiselinks=true    % forces commands to reflect the real height of the link 
   ,breaklinks=true    % Allows link text to break across lines
   ,pageanchor=true    % Determines whether every page is given an implicit
   ,plainpages=false   % Forces page anchors to be named by the arabic
   ,linktocpage=true   % make page number, not text, be link on TOC, LOF and LOT
   ,colorlinks=true    % Colors the text of links and anchors.
   ,linkcolor  =pdflinkcolor   % Color for normal internal links.
   ,anchorcolor=pdfanchorcolor % Color for anchor text.
   ,citecolor  =pdfcitecolor   % Color for bibliographical citations in text.
   ,filecolor  =pdffilecolor   % Color for URLs which open local files.
   ,menucolor  =pdfmenucolor   % Color for Acrobat menu items.
   ,runcolor   =pdfruncolor    % Color for run links (launch annotations).
   ,urlcolor   =pdfurlcolor    % color magenta Color for linked URLs.
   ,bookmarksopen=true     % If Acrobat bookmarks are requested, show them
   ,bookmarksopenlevel=2   % level (\maxdimen) to which bookmarks are open
   ,bookmarksnumbered=true %
   ,bookmarkstype=toc      %
   ,pdfpagemode=UseOutlines %
   ,pdfstartpage=1         % Determines on which page the PDF file is opened.
   ,pdfstartview=FitH      % Set the startup page view
   ,pdfremotestartview=Fit % Set the startup page view of remote PDF files
   ,pdfcenterwindow=false  %
   ,pdffitwindow=false     % resize document window to fit document size
   ,pdfnewwindow=false     % make links that open another PDF file 
   ,pdfdisplaydoctitle=true  % display document title instead of file name 
} % end: hypersetup
%
%%% color setup
% table colors 
\colorlet{tablebodycolor}{white!100}
\colorlet{tablerowcolor}{gray!10}
\colorlet{tableheadcolor}{gray!25}
%%% Set document layout / variables
\setlength{\parindent}{0pt}
\setlength{\parskip}{0.5\baselineskip}
\setcounter{secnumdepth}{2}
\setcounter{tocdepth}{2}
%%% doc commands
\newcommand{\package}[1]{\texttt{#1}}
\newcommand{\option}[1]{\texttt{#1}}
\newcommand{\parameter}[1]{\texttt{#1}}
\renewcommand\arg[1]{\meta{\normalfont\slshape#1}}
% 
\newcommand{\latex}{\LaTeX}
\newcommand{\Default}[1]{\par Default: \texttt{#1} \par}
\newcommand{\Example}[1]{\par Example: \texttt{#1} \par}

\newenvironment{Optionlist}{%
\begin{flushright}%
%  Style  changes
\small\renewcommand{\arraystretch}{1.4}%
%  tabu
\begin{tabu} to 1.0\linewidth  {>{\ttfamily}l<{\normalfont}X[1,l]}%
}{%
\end{tabu}%
\end{flushright}%
}


\listfiles

\begin{document}
  \DocInput{templatedemo.dtx}
  \PrintChanges
  \PrintIndex
\end{document}
%</driver>
% \fi
%
% \CheckSum{226}
%
% \CharacterTable
%  {Upper-case    \A\B\C\D\E\F\G\H\I\J\K\L\M\N\O\P\Q\R\S\T\U\V\W\X\Y\Z
%   Lower-case    \a\b\c\d\e\f\g\h\i\j\k\l\m\n\o\p\q\r\s\t\u\v\w\x\y\z
%   Digits        \0\1\2\3\4\5\6\7\8\9
%   Exclamation   \!     Double quote  \"     Hash (number) \#
%   Dollar        \$     Percent       \%     Ampersand     \&
%   Acute accent  \'     Left paren    \(     Right paren   \)
%   Asterisk      \*     Plus          \+     Comma         \,
%   Minus         \-     Point         \.     Solidus       \/
%   Colon         \:     Semicolon     \;     Less than     \<
%   Equals        \=     Greater than  \>     Question mark \?
%   Commercial at \@     Left bracket  \[     Backslash     \\
%   Right bracket \]     Circumflex    \^     Underscore    \_
%   Grave accent  \`     Left brace    \{     Vertical bar  \|
%   Right brace   \}     Tilde         \~}
%
%
% \changes{v0.1}{2012/05/20}{initial version (converted to dtx file)}
%
% \DoNotIndex{\newcommand,\newenvironment}
%
% \providecommand*{\url}{\texttt}
% \GetFileInfo{templatedemo.dtx}
% \title{The \textsf{templatedemo} package}
% \author{\authortext \\ \url{\emailtext}}
% \date{\fileversion~from \filedate}
%
% \maketitle
% \section{Introduction}
% In order to demonstrate \latex{} code it is very useful to have the code and
% the resulting output together in the same document. \demopackage{\packagetext} 
% is a package that provides configurable tools to print out \latex{} code and the
% resulting output in the same document.
%
% The difference to other similar packages is the support of
% verbatim material inside a conditional sequence and the consequence that each 
% code must be written to an external file.
%
% The commands provided by this package are based on the packages 
% \package{listings}, \package{mdframed} and \package{filecontents}. 
%
% \section{Basic example}
% \label{sec:example:basic}
% Below is a principle example,
% which demonstrates the usage of the package by showing some commands of a
% package (\demopackage{soul}) with the resulting code side by side:
%
% \iffalse
%<*example>
% \fi
\begin{filecontents*}{\democodefile}
\so{letterspacing}, \\
\ul{underlining},   \\
\st{overstriking}   \\
and \hl{highlighting}. 
\end{filecontents*}
% \iffalse
%</example>
% \fi
%
% \PrintDemo{style=parallel}
%
% which is created with the code of this package using \cs{ifcsdef} from etoolbox
% to test if the code will run through or fail because of unkown commands.
%
% \iffalse
%<*example>
% \fi
\begin{lstlisting}[style=demostyle]
\begin{filecontents*}{\democodefile}
\so{letterspacing}, \\
\ul{underlining},   \\
\st{overstriking}   \\
and \hl{highlighting}. 
\end{filecontents*}

\ifcsdef{so}{%
%
\PrintDemo{style=parallel}
%
}{% 
  \DemoError{Command \democs{so} of package% 
  \demopackage{soul} not available.
  Probably the package was not loaded.
  }
}%
\end{lstlisting}
% \iffalse
%</example>
% \fi
%
% \section{Usage}
%
% \subsection{Define code}
% This package requires the example code to be written to an external file. The
% filename is saved in the command sequence \cs{democodefile}. The output is done
% with the \demoenv{filecontents} environment:
%
% \iffalse
%<*example>
% \fi
\begin{lstlisting}[style=demostyle]
\begin{filecontents*}{\democodefile}
... code ...
\end{filecontents*}
\end{lstlisting}
% \iffalse
%</example>
% \fi
%
% The requirement for an external file originates from the problem that verbatim
% content cannot be saved in normal tex macros since the line breaks and white
% spaces get lost. Furthermore, any solution which would solve this problem would
% fails finally because saving such contend cannot be included in conditional
% code statements\footnote{See discussion on:
%  \url{http://tex.stackexchange.com/questions/29256/}}.
%
% The content in the file defined by \cs{democodefile} is further read for the
% printing of the code and the corresponding output.
%
% \subsection{Print code and result}
%
% \DescribeMacro{\PrintDemo\{style=\arg{option}\}}%
% is the macro for the output of code and result. 
% The layout of both is defined with the style option. 
% The following options are possible
%
% \begin{Optionlist}
%  parallel &  code and result side by side.  \\
%  stacked  &  (default) code and result with 100\,\% text width stacked.\\
%  lines    &  like \texttt{stacked}, but with lines on 
%              top and bottom of the result instead of a surrounding box. \\
%  none     &  like \texttt{stacked}, but with nothing around the result. \\
%  page     &  result on a single page.\\
% \end{Optionlist}
%
% You should use \option{parallel} for small examples and \option{stacked}
% otherwise. The options \option{lines} and \option{none} is primarily 
% for those cases where a surrounding box is disturbing or impossible. 
% The latter occurs for example in cases where content is written across 
% the text width boundaries. The option \option{page} is obviously for 
% those cases where the output is very large or written to another page anyway.
%
% \subsubsection{Examples}
% The following code is used in the examples
% \iffalse
%<*example>
% \fi
\begin{lstlisting}[style=demostyle]
\begin{filecontents*}{\democodefile}
This code shows some basic math: $a^2 + b^2 = c^2$.
\end{filecontents*}
\end{lstlisting}
%
\begin{filecontents*}{\democodefile}
This code shows some basic math: $a^2 + b^2 = c^2$.
\end{filecontents*}
% \iffalse
%</example>
% \fi
%
% \begin{itemize}
% \item \cmd{\PrintDemo\{style=parallel\}}
% \PrintDemo{style=parallel}
%
% \item \cmd{\PrintDemo\{style=stacked\}}
% \PrintDemo{style=stacked}
%
% \item \cmd{\PrintDemo\{style=lines\}}
% \PrintDemo{style=lines}
%
% \item \cmd{\PrintDemo\{style=none\}}
% \PrintDemo{style=none}
% \end{itemize}
%
% The prefix text for code and result and the filename can be changed. The
% commands are introduced in section~\ref{sec:setup}. Commands for the output of
% content such as the name of of a package, a command, an environment or a
% generalized error message are introduced in section~\ref{sec:commands:output}.
%
% \subsection{Setup}
% \label{sec:setup}
%
% The following commands define the name for the temporary file or the strings
% used for the printing of code or results. Use \cs{renewcommand} to change the
% definitions.
%
% \DescribeMacro{\democodefile}%
% Filename for the temporary file required for code and results printing. 
% \Default{democode}
%
% \DescribeMacro{\democodeprefix}%
% Prefix text for output of code.
% \Default{Code: }
%
% \DescribeMacro{\demoresultprefix}%
% Prefix text for output of the result. 
% \Default{Result: }
%
% \subsection{Output commands}
% \label{sec:commands:output}
%
% The following commands are provided for the user to print out and format
% commands, environments, packages and errors.
%
% \DescribeMacro{\democs}%
% shortcut for \cs{democommand}
% \Example{\democs{foo}}
%
% \DescribeMacro{\democommand}%
% formats the given command sequence name. Note that a parameter
% \parameter{foo} is printed as \cs{foo}.
% \Example{\democommand{foo}}
%
% \DescribeMacro{\demopackage}%
% formats the given packagename.
% \Example{\demopackage{foo}}
%
% \DescribeMacro{\demoenv}%
% formats the given environment name.
% \Example{\demoenv{foo}}
%
% \DescribeMacro{\DemoError}%
% prints out the given error message
% \Example{\DemoError{foo}}
%
%
% \StopEventually{}
%
% \section{Implementation}
%
% \iffalse
%<*templatedemo.sty>
% \fi
%
%    \begin{macrocode}
\NeedsTeXFormat{LaTeX2e}
\ProvidesPackage{templatedemo}[2012/05/20 v0.1 typeset code and resulting output]
\RequirePackage{listings}
% This code needs to be executed at the beginning
% of the document because some packages (eg. xcolor)
% could lead to option clashes otherwise
%
\AtBeginDocument{%
%    \end{macrocode}
% \subsection{Preamble}
% \subsubsection{Packages}
% 
%    \begin{macrocode}
%% Programming
\RequirePackage{xspace}
\RequirePackage{etoolbox}
%% Write contents to files
\RequirePackage{filecontents}
%% Package for colors
\RequirePackage{xcolor}
%% Packages for frames
\RequirePackage{mdframed}
\RequirePackage{framed}
%    \end{macrocode}
%
% \subsubsection{Colors}
%
%    \begin{macrocode}
%% Colors
\colorlet{demo@stringcolor}{green!40!black!100}
\colorlet{demo@commentcolor}{green!50!black!100}
\colorlet{demo@numbercolor}{white!50!black!100}
\colorlet{demo@codebackcolor}{white!95!black!100}
\colorlet{demo@resultbackcolor}{white}
\definecolor{demo@keywordcolor}{rgb}{0,0.47,0.80}
\definecolor{demo@rulecolor}{rgb}{0,0.4,0.5}
\definecolor{demo@code@rulecolor}{rgb}{0.5,0.5,0.5}
%    \end{macrocode}
%
% \subsection{Commands}
%
%    \begin{macrocode}
%% === Simple Commands ===============================
%    \end{macrocode}
%
% \begin{macro}{\bs}
% Shortcut for \cs{textbackslash}.
%    \begin{macrocode}
\newcommand{\bs}{\textbackslash}
%    \end{macrocode}
% \end{macro}
%
% \begin{macro}{\democs}
% Shortcut for \cs{democommand}.
%    \begin{macrocode}
\newcommand{\democs}[1]{\democommand{#1}}
%    \end{macrocode}
% \end{macro}
%
% \begin{macro}{\democommand}
% Prints the argument with backslash in typewriter font. 
% This is meant to be used for \latex{} commands.
%    \begin{macrocode}
\newcommand{\democommand}[1]{\texttt{\bs{}#1}}
%    \end{macrocode}
% \end{macro}
%
% \begin{macro}{\demoenv}
% Format an environment in typewriter font. 
%    \begin{macrocode}
\newcommand{\demoenv}[1]{\texttt{#1}}
%    \end{macrocode}
% \end{macro}
%
% \begin{macro}{\demopackage}
% Format an package in typewriter font. 
%    \begin{macrocode}
\newcommand{\demopackage}[1]{\texttt{#1}}
%    \end{macrocode}
% \end{macro}
%
% \begin{macro}{\democodefile}
% Saves the filename for temporary file output.
%    \begin{macrocode}
\newcommand{\democodefile}{democode}
%    \end{macrocode}
% \end{macro}
%
% \begin{macro}{\democodeprefix}
% Prefix text for code output.
%    \begin{macrocode}
\newcommand{\democodeprefix}{Code: }
%    \end{macrocode}
% \end{macro}
%
% \begin{macro}{\demoresultprefix}
% Prefix text for result output.
%    \begin{macrocode}
\newcommand{\demoresultprefix}{\noindent Result:}
%    \end{macrocode}
% \end{macro}
%
% \begin{macro}{\DemoError}
% Output and formatting of error messages.
%    \begin{macrocode}
%% Print Error
\newcommand{\DemoError}[1]{
  \ifcsdef{textcolor}
    {\textcolor{red}{Error:~}}
    {Error:~}
  #1 \par\noindent
}
%    \end{macrocode}
% \end{macro}
%
% \subsection{Define keys}
%
%    \begin{macrocode}
%% === Define Keys ===================================
\RequirePackage{kvoptions-patch}
\RequirePackage{kvoptions}  % options
\RequirePackage{pdftexcmds} % string comparison
\SetupKeyvalOptions{family=demo,prefix=demo@}
%    \end{macrocode}
% Define default option for style key: \emph{stacked}
%    \begin{macrocode}
\DeclareStringOption[stacked]{style}
\ProcessKeyvalOptions{demo}
%    \end{macrocode}
%
% \begin{macro}{\PrintDemoUsingKeys}
% Evaluate key and execute corresponding commands
%    \begin{macrocode}
\newcommand{\PrintDemoUsingKeys}{%
  \ifnum\pdf@strcmp{\demo@style}{parallel}=0%
    \PrintCodeAndResultsParallel%
  \else\ifnum\pdf@strcmp{\demo@style}{stacked}=0%
    \PrintCodeAndResultsStacked%
  \else\ifnum\pdf@strcmp{\demo@style}{lines}=0%
    \PrintCodeAndResultsStackedLines%
  \else\ifnum\pdf@strcmp{\demo@style}{page}=0%
    \PrintCodeAndResultsPage%
  \else\ifnum\pdf@strcmp{\demo@style}{none}=0%
    \PrintCodeAndResultsNone%
  \else%
     \PackageError{templatedemo}{%
       \MessageBreak%
       value >\tplbugs@style< unkown \MessageBreak%
     }{}%
  \fi\fi\fi\fi\fi%
}%
%    \end{macrocode}
% \end{macro}
%
%
% \begin{macro}{\PrintDemo}
% Print code and result using the key-value syntax
%    \begin{macrocode}
\newcommand{\PrintDemo}[1]{%
\begingroup
  \setkeys{demo}{#1}%
  \PrintDemoUsingKeys
\endgroup
}    
%    \end{macrocode}
% \end{macro}
%
% \subsection{listings package style}
%    \begin{macrocode}
%% === Listings style ================================
\lstdefinestyle{demostyle}{
  columns=flexible,
  keepspaces=true,    
  numberstyle=\tiny\color{demo@numbercolor},%
  basicstyle=\small\ttfamily,%
  numbers=none,%
  stepnumber=1,%
  numbersep=5pt,%
  tabsize=2,%
  extendedchars=true,%
  breaklines=true,%
  stringstyle=\color{demo@stringcolor}, %
  keywordstyle=\color{demo@keywordcolor}, %
  commentstyle=\color{demo@commentcolor}, %
  showspaces=false,%
  showtabs=false,%
  showstringspaces=false,%
  frame=single,%
  backgroundcolor=\color{demo@codebackcolor},%
  rulecolor=\color{demo@code@rulecolor},%
  language = [LaTeX]TeX,%
  moretexcs={maketitle,tableofcontents,subsection,text,includegraphics,
    chapter,tableofcontents,section,subsection,subsubsection,paragraph,si,SI,
    textmu,newcolumntype,rowcolor,rowcolors,bottomrule,toprule,midrule,
    mainmatter,frontmatter,geometry,KOMAoptions,enquote,blockquote,ding,mathds,
    ifcsdef,mathcal,underset,operatorname,sfrac,Bra,Ket,Braket,setlength,
    subcaption,sisetup,rowfont,taburowcolors,theadstart,tbody,tsubheadstart,
    tsubhead,tend,unit,unitfrac,micro,gls,printglossary,glsadd,newglossaryentry,
    lettrine,mdfsetup,democodefile,demopackage,democs,democommand,demoenv,
    DemoError},
}
\lstloadlanguages{[LaTeX]TeX}
%    \end{macrocode}
% Create a new environment for listings
%    \begin{macrocode}
\lstnewenvironment{latexcode}{%
\lstset{style=demostyle, inputencoding=latin1, escapeinside=!!}%
\democodeprefix
}{}
%    \end{macrocode}
%
% \subsection{mdframed package style}
%
%    \begin{macrocode}
%% === Mdframed style ================================
\mdfdefinestyle{DemoStyleFrames}{
  linecolor=demo@rulecolor,%
  linewidth=0.8pt,
  skipabove=0pt,
  skipbelow=0.5\baselineskip,
  leftmargin =-3.5pt,
  rightmargin=-3.5pt,
  innerleftmargin=3pt,
  innerrightmargin=3pt,
}% 
%    \end{macrocode}
%
% \subsection{Commands for the formatting}
%
% \begin{macro}{\preResultSkip}
% Default skip at the beginning of a result
%    \begin{macrocode}
%% === Formating commands ============================
\newcommand{\preResultSkip}{\vspace*{-0.5\baselineskip}}
%    \end{macrocode}
% \end{macro}
%
%
% \begin{environment}{latexresult}
% Environment to print the result in a box
%    \begin{macrocode}
\newenvironment{latexresult}{%
\demoresultprefix
\nopagebreak[4]
\preResultSkip
\mdframed[%
  style=DemoStyleFrames,
  backgroundcolor=demo@resultbackcolor,%
  usetwoside=false,
]%
}{
\endmdframed
\noindent
}
%    \end{macrocode}
% \end{environment}
%
%
% \begin{macro}{\resultline}
%% Single Line for results
%    \begin{macrocode}
\newcommand{\resultline}{%
\nopagebreak[4]
%% Insert single line
\mdframed[%
  style=DemoStyleFrames,
  skipabove=3pt,
  skipbelow=3pt,
  topline=true,bottomline=false,leftline=false,rightline=false,
  backgroundcolor=white,%
]\mbox{}\endmdframed
\nopagebreak[4]
}
%    \end{macrocode}
% \end{macro}
%
% % \subsection{Low level commands for printing of code and result}
%
% \begin{macro}{\printlatexcode}
% Prints the code using \cs{lstinputlisting}
%    \begin{macrocode}
%% === Output commands ===============================
%% Print Code with prefix
\newcommand{\printlatexcode}{%
\democodeprefix
\lstinputlisting[style=demostyle,nolol=true]{\democodefile}%
}%
%    \end{macrocode}
% \end{macro}
%
%
% \begin{macro}{\printlatexresult}
% Prints the result enclosed in the \demoenv{latexresult} environment.
% The evaluation of the code is simply achieved by loading the file with 
% \cs{input}.
%    \begin{macrocode}
%% Print Result with standard box
\newcommand{\printlatexresult}{%
\begin{latexresult}%
\IfFileExists{\democodefile}{\input{\democodefile}}{}%
\end{latexresult}%
}%
%    \end{macrocode}
% \end{macro}
%
%
% \begin{macro}{\printlatexresultlines}
% Like \cs{printlatexresult} but with lines above and below instead of a
% surrounding box.
%    \begin{macrocode}
%% Print result with lines 
\newcommand{\printlatexresultlines}{%
\demoresultprefix
\nopagebreak[4] \resultline \nopagebreak[4]
\IfFileExists{\democodefile}{\input{\democodefile}}{}%
\nopagebreak[4] \resultline \nopagebreak[4]
}%
%    \end{macrocode}
% \end{macro}
%
% \subsection{Output of code and result}
%
% \begin{macro}{\PrintCodeAndResultsParallel}
%    \begin{macrocode}
%% === Output commands for code and result ===========
\newcommand{\PrintCodeAndResultsParallel}{%
\nopagebreak[4]
\vspace*{0.5em}\par\noindent
\begin{minipage}[t]{0.48\linewidth}
\printlatexcode
\end{minipage} \hfill
\begin{minipage}[t]{0.48\linewidth}
\printlatexresult
\end{minipage}
\par\noindent
}
%    \end{macrocode}
% \end{macro}
%
%
% \begin{macro}{\PrintCodeAndResultsStacked}
%    \begin{macrocode}
\newcommand{\PrintCodeAndResultsStacked}{%
\nopagebreak[4]
\vspace*{0.5em}\par\noindent
\printlatexcode%
\printlatexresult%
\par\noindent
}%
%    \end{macrocode}
% \end{macro}
%
%
% \begin{macro}{\PrintCodeAndResultsStackedLines}
%    \begin{macrocode}
\newcommand{\PrintCodeAndResultsStackedLines}{%
\nopagebreak[4]
\vspace*{0.5em}\par\noindent
\printlatexcode%
\printlatexresultlines%
\vspace*{0.5em}\par\noindent
}%
%    \end{macrocode}
% \end{macro}
%
%
% \begin{macro}{\PrintCodeAndResultsNone}
%    \begin{macrocode}
\newcommand{\PrintCodeAndResultsNone}{%
\nopagebreak[4]
\vspace*{0.5em}\par\noindent
\printlatexcode%
%    \end{macrocode}
%
%    \begin{macrocode}
\demoresultprefix
\nopagebreak[4]
\par\noindent
\IfFileExists{\democodefile}{\input{\democodefile}}{}%
%    \end{macrocode}
%
%    \begin{macrocode}
\vspace*{0.5em}\par\noindent
}%
%    \end{macrocode}
% \end{macro}
%
%
% \begin{macro}{\PrintCodeAndResultsPage}
%    \begin{macrocode}
\newcommand{\PrintCodeAndResultsPage}{%
\nopagebreak[4]
\vspace*{0.5em}\par\noindent
\printlatexcode%
\demoresultprefix: Shown on the following page.
\newpage
\IfFileExists{\democodefile}{\input{\democodefile}}{}%
\newpage
}%
%    \end{macrocode}
%
%    \begin{macrocode}
} % end of \AtBeginDocument
%    \end{macrocode}
% \end{macro}
%
%
% \iffalse
%</templatedemo.sty>
% \fi
%
% \Finale
\endinput
