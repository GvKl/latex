\DefineTemplateSection[true]{PackagesBase}
\DefineTemplateSection[true]{PackagesBugfixes}
\DefineTemplateSection[true]{PackagesFonts}
\DefineTemplateSection[true]{PackagesDiagrams}
\DefineTemplateSection[true]{PackagesMath}
\DefineTemplateSection[true]{PackagesScience}
\DefineTemplateSection[true]{PackagesSymbols}
\DefineTemplateSection[true]{PackagesTables}
\DefineTemplateSection[true]{PackagesText}
\DefineTemplateSection[true]{PackagesQuotes}
\DefineTemplateSection[true]{PackagesCitation}
\DefineTemplateSection[true]{PackagesFigures}
\DefineTemplateSection[true]{PackagesCaptions}
\DefineTemplateSection[true]{PackagesIndexes}
\DefineTemplateSection[true]{PackagesMisc}
\DefineTemplateSection[true]{PackagesVerbatim}
\DefineTemplateSection[true]{PackagesFancy}
\DefineTemplateSection[true]{PackagesLayout}
\DefineTemplateSection[true]{PackagesHeadFoot}
\DefineTemplateSection[true]{PackagesHeadings}
\DefineTemplateSection[true]{PackagesTOC}
\DefineTemplateSection[true]{PackagesPDF}
\DefineTemplateSection[true]{PackagesAdditional}

% ~~~~~~~~~~~~~~~~~~~~~~~~~~~~~~~~~~~~~~~~~~~~~~~~~~~~~~~~~~~~~~~~~~~~~~~~
% These packages must be loaded before all others
% (primarily because they are required by other packages)
% ~~~~~~~~~~~~~~~~~~~~~~~~~~~~~~~~~~~~~~~~~~~~~~~~~~~~~~~~~~~~~~~~~~~~~~~~
\BeginTemplateSection{PackagesBase}

% Description: Calculation with LaTeX 
% Doc: calc.pdf
\usepackage{calc}

% Description: Multi Language support for LaTeX
% Doc: babel.pdf
\usepackage{babel}
% Description: support automatic translations
% Doc: beameruserguide.pdf
\usepackage{translator}


% Description: Color support with color mixing modells
% Doc: xcolor.pdf
\usepackage[
  dvipsnames, % Load a set of predefined colors 
  table,      % Load the colortbl package
  % fixpdftex,  % Load the pdfcolmk package (may be problematic)
  hyperref,   % Support  the  hyperref  package
  fixinclude, %Prevent dvips color reset before .eps file inclusion
]{xcolor}

% Description: Support for graphics in LaTeX
% Doc: grfguide.pdf
\usepackage[%
  %final,
  %draft % do not include images (faster)
]{graphicx}


% Description: If an eps image is detected, epstopdf is automatically 
%              called to convert it to pdf format.
% Requires: graphicx loaded
% Doc: epstopdf.pdf
\IfPackageLoaded{graphicx}{%
  \usepackage{epstopdf}
}


% Description:  environments for setting ragged text 
%               which allow hyphenation.
% Provides: \Centering, \RaggedLeft, and \RaggedRight, ... 
% Doc: ragged2e.pdf
\usepackage{ragged2e}

\EndTemplateSection{PackagesBase}
% ~~~~~~~~~~~~~~~~~~~~~~~~~~~~~~~~~~~~~~~~~~~~~~~~~~~~~~~~~~~~~~~~~~~~~~~~
% latex bug fixing packages
% ~~~~~~~~~~~~~~~~~~~~~~~~~~~~~~~~~~~~~~~~~~~~~~~~~~~~~~~~~~~~~~~~~~~~~~~~
\BeginTemplateSection{PackagesBugfixes}

% Description: Fix known LaTeX2e bugs
% Doc: fixltx2e.pdf
\usepackage{fixltx2e}

% Description: This package implements a workaround for the LaTeX bug that
%              marginpars sometimes appear on the wrong margin.
% \usepackage{mparhack}
% BUG: in some case this causes an error in the index together with package
%      pdfpages the reason is unkown. Therefore I recommend to use the
%      margins of marginnote
% incompatible: marginfix

% Description: marginnote allows a margin note, where \marginpar fails 
% Doc: marginnote.pdf
\usepackage{marginnote}

% Description: Redefines implementations of 
%              packages float, hyperref and listings
% Doc: scrhack.pdf
\usepackage{scrhack}

%% Description: changes the \marginpar commands, such
%%              that long margin notes work.
%% Doc: marginfix.pdf (TODO: why not used)
% \usepackage{marginfix}

% Description: Used to define commands that don't eat spaces.
% Doc: xspace.pdf
\RequirePackage{xspace}

\EndTemplateSection{PackagesBugfixes}
% ~~~~~~~~~~~~~~~~~~~~~~~~~~~~~~~~~~~~~~~~~~~~~~~~~~~~~~~~~~~~~~~~~~~~~~~~
% Fonts
% ~~~~~~~~~~~~~~~~~~~~~~~~~~~~~~~~~~~~~~~~~~~~~~~~~~~~~~~~~~~~~~~~~~~~~~~~

\BeginTemplateSection{PackagesFonts}

%% Description: Set the font size relative to the current font size
%% Doc: relsize-doc.pdf
\usepackage{relsize}

\EndTemplateSection{PackagesFonts}

% ~~~~~~~~~~~~~~~~~~~~~~~~~~~~~~~~~~~~~~~~~~~~~~~~~~~~~~~~~~~~~~~~~~~~~~~~
% Math Packages
% ~~~~~~~~~~~~~~~~~~~~~~~~~~~~~~~~~~~~~~~~~~~~~~~~~~~~~~~~~~~~~~~~~~~~~~~~
\BeginTemplateSection{PackagesMath}


% Description: basic math package
% Doc: amsldoc.pdf
\usepackage[
   centertags, % (default) center tags vertically
   %tbtags,    % 'Top-or-bottom tags': For a split equation, place equation
               % numbers level with the last (resp. first) line, if numbers
               % are on the right (resp. left).
   sumlimits,  %(default) Place the subscripts and superscripts of summation
               % symbols above and below
   %nosumlimits, % Always place the subscripts and superscripts of
                 % summation-type symbols to the side, even in displayed
                 % equations.
   intlimits,  % Like sumlimits, but for integral symbols.
   %nointlimits, % (default) Opposite of intlimits.
   namelimits, % (default) Like sumlimits, but for certain 'operator names'
               % such as det, inf, lim, max, min, that traditionally have
               % subscripts placed underneath when they occur in a displayed
               % equation.
   %nonamelimits, % Opposite of namelimits.
   %leqno,     % Place equation numbers on the left.
   %reqno,     % Place equation numbers on the right.
   fleqn,      % Position equations at a fixed indent from the left margin
               % rather than centered in the text column.
]{amsmath} %

\IfPackageLoaded{amsmath}{

% Description: The mathtools package is an extension package to amsmath. 
%              Furthermore it  correct various bugs
% Doc: mathtools.pdf
\usepackage[fixamsmath,disallowspaces]{mathtools}

% Description: Inhibits the usage of plain TeX and 
%              of standard LaTeX math environments
% Doc: onlyamsmath.pdf
\usepackage[
  all,
  % warning
  error
]{onlyamsmath}

} % end: IfPackageLoaded{amsmath}

% Description: Macros for Dirac bra–ket notation and sets.
% Doc: braket.pdf
\usepackage{braket}

% Description: strike out arguments in math mode
% Doc: cancel.sty
\usepackage{cancel}

%% Description: Emphasize equations
%% Doc: empheq.pdf
\usepackage{empheq}  % Hervorheben

% Description: scales math mode output in all environments correct
% Doc: Mathmode.pdf
\IfPackageNotLoaded{MnSymbol}{
   \usepackage{exscale} 
}

% Description: fixes for the default Computer Modern math fonts
% Doc: fixmath.pdf
\IfPackageLoaded{lmodern}{%
  \usepackage{fixmath}
}

% Description: Enables the correct use of the comma as 
%              a decimal separator in math mode
% Doc: icomma.pdf
\usepackage{icomma}

% Description: LaTeX 3 Package for nice inline fractions
% Provides: \sfrac{1}{2}
% Replaces: nicefrac
% Doc: xfrac.pdf 
\usepackage{xfrac} 

\EndTemplateSection{PackagesMath}
% ~~~~~~~~~~~~~~~~~~~~~~~~~~~~~~~~~~~~~~~~~~~~~~~~~~~~~~~~~~~~~~~~~~~~~~~~
% diagrams
% ~~~~~~~~~~~~~~~~~~~~~~~~~~~~~~~~~~~~~~~~~~~~~~~~~~~~~~~~~~~~~~~~~~~~~~~~
\BeginTemplateSection{PackagesDiagrams}

\usepackage{pgf}
\usepackage{tikz}
\IfPackageLoaded{pgf}{%
% \usepgflibrary{arrows}
}

\IfPackageLoaded{tikz}{%
%%% Chapter numbers according to 
%%% package version 2.10
%
%%% 12. Package, Environments, Scopes, and Styles
\usetikzlibrary{scopes}         % Shorthand for Scope Environments
\usetikzlibrary{intersections}  % Intersections of Arbitrary Paths
%%% 13. Specifying Coordinate
\usetikzlibrary{calc}           % Coordinate Calculations
%%% 14. Syntax for Path Specifications
%%% 15. Actions on Path
%%% 16. Nodes and Edge
\usetikzlibrary{positioning}    % Advanced Placement Options
%%% 17. Matrices and Alignment
%%% 18. Making Trees Grow
%%% 19. Plots of Function
%%% 20. Transparency
%%% 21. Decorated Path
\usetikzlibrary{decorations}
%%% 22. Transformation
%%% 23. Arrow Tip Library
\usetikzlibrary{arrows}
%%% 24. Automata Drawing Library
\usetikzlibrary{automata}
%%% 25. Background Library
\usetikzlibrary{backgrounds}
%%% 26. Calc Library -> see 13.
%%% 27. Calendar Library
%\usetikzlibrary{calendar}
%%% 28. Chains
\usetikzlibrary{chains}
%%% 29. Circuit Libraries
\usetikzlibrary{circuits}
\usetikzlibrary{circuits.logic.IEC}
\usetikzlibrary{circuits.ee.IEC}
%\usetikzlibrary{circuits.logic.US}
%%% 30. Decoration Library -> see 21.
%%% 31. Entity-Relationship Diagram Drawing Library
\usetikzlibrary{er}
%%% 32. Externalization Library
%\usetikzlibrary{external} % error: no room for a new \write
%%% 33. Fading Library
\usetikzlibrary{fadings}
%%% 34. Fitting Library
\usetikzlibrary{fit}
%%% 35. Fixed Point Arithmetic Library
\usetikzlibrary{fixedpointarithmetic}
%%% 36. Floating Point Unit Library
\usetikzlibrary{fpu}
%%% 37. Lindenmayer System Drawing Library
%\usetikzlibrary{lindenmayersystems}
%%% 38. Matrix Library
\usetikzlibrary{matrix}
%%% 39. Mindmap Drawing Library
%\usetikzlibrary{mindmap}        % Mindmap Drawing Library
%%% 40. Paper Folding Diagrams Library
%\usetikzlibrary{folding}
%%% 41. Pattern Library
\usetikzlibrary{patterns}       % Pattern Library
%%% 42. Petri-Net Drawing Library
%\usetikzlibrary{petri}
%%% 43. Plot Handler Library (loaded autom.)
\usetikzlibrary{plothandlers}
%%% 44. Plot Mark Library
\usetikzlibrary{plotmarks}
%%% 45. Profiler Library
%%% 46. Shadings Library
\usetikzlibrary{shadings}
%%% 47. Shadow Library
\usetikzlibrary{shadows}
%%% 48. Shape Library
\usetikzlibrary{shapes.geometric}
\usetikzlibrary{shapes.symbols}
\usetikzlibrary{shapes.multipart}
\usetikzlibrary{shapes.callouts}
\usetikzlibrary{shapes.misc}
%%% 49. Spy Library: Magnifying Parts of Pictures
%%% 50. SVG-Path Library
%%% 51. To Path Library (loaded autom.)
%%% 52. Through Library
%%% 53 Tree Library
%\usetikzlibrary{trees}
%%% 54 Turtle Graphics Library
%\usetikzlibrary{turtle}
}


\usepackage{pgfplots}
\usepackage{pgfplotstable}
\usetikzlibrary{pgfplots.patchplots}
\usetikzlibrary{pgfplots.dateplot}
\usetikzlibrary{pgfplots.colormaps}
\usetikzlibrary{pgfplots.groupplots}
\usetikzlibrary{pgfplots.polar}
\usetikzlibrary{pgfplots.units}

\EndTemplateSection{PackagesDiagrams}

% ~~~~~~~~~~~~~~~~~~~~~~~~~~~~~~~~~~~~~~~~~~~~~~~~~~~~~~~~~~~~~~~~~~~~~~~~
% science packages
% ~~~~~~~~~~~~~~~~~~~~~~~~~~~~~~~~~~~~~~~~~~~~~~~~~~~~~~~~~~~~~~~~~~~~~~~~
\BeginTemplateSection{PackagesScience}
 
% Description: upright symbols from euler package
%              [Euler] or Adobe Symbols [Symbol]
% Provides:    \upmu
% Doc: upgreek.pdf
%\usepackage[Symbolsmallscale]{upgreek} 
% --> Use only if the original font does not provide
%     the necessary upright symbols

% Description: Commands/symbols for both math and text mode
% Provides:    \degree, \celsius, \perthousand, \ohm, \micro
% Incompatible: siunitx
% Requires: Command \upmu
% \IfDefined{upmu}{\usepackage[upmu]{gensymb}}

% Description:  package for setting units in a 
%               typographically correct way.
% Incompatible: siunitx
%\usepackage{units}

% Description: siunitx aims to provide a unified method to
%              typeset numbers and units correctly and easily.
% Incompatible: gensymb, units
\IfPackagesNotLoaded{gensymb, units}{
  \usepackage{siunitx}
}

\EndTemplateSection{PackagesScience}

% ~~~~~~~~~~~~~~~~~~~~~~~~~~~~~~~~~~~~~~~~~~~~~~~~~~~~~~~~~~~~~~~~~~~~~~~~
% Symbole
% ~~~~~~~~~~~~~~~~~~~~~~~~~~~~~~~~~~~~~~~~~~~~~~~~~~~~~~~~~~~~~~~~~~~~~~~~
\BeginTemplateSection{PackagesSymbols}
%%% General Doc: symbols-a4.pdf
%
%% Math symbols
\IfPackagesNotLoaded{mathdesign,MnSymbol,MdSymbol}{
  \usepackage{dsfont}   %% Double Stroke Fonts
}{}
% Futher Math symbols
\IfPackagesNotLoaded{MnSymbol,MdSymbol}{
  \usepackage{esint} % generate missing integrals for lmodern
  % provides further symbols of the Text Companion (TC) fonts
  % such as \tcmu, \tcperthousand, \tcdegree
  \usepackage{mathcomp} 
  \usepackage[mathcal]{euscript} %% adds euler mathcal font
}{}

%\usepackage[integrals]{wasysym}

%% Common Symbols
\usepackage{pifont}   %% ZapfDingbats

\EndTemplateSection{PackagesSymbols}

% ~~~~~~~~~~~~~~~~~~~~~~~~~~~~~~~~~~~~~~~~~~~~~~~~~~~~~~~~~~~~~~~~~~~~~~~~
% Tables (Tabular)
% ~~~~~~~~~~~~~~~~~~~~~~~~~~~~~~~~~~~~~~~~~~~~~~~~~~~~~~~~~~~~~~~~~~~~~~~~
\BeginTemplateSection{PackagesTables}

% Description:  some additional commands to enhance
%               the quality of tables
% Provides:     \toprule, \midrule, \bottomrule, \cmidrule
% Doc: booktabs.pdf
\usepackage{booktabs}

% Description: extends the standard tabular environment with cells
%              spanning over multiple rows.
% Doc: multirow.pdf
\usepackage{multirow, bigstrut}

% Description: Table spanning over many pages (from longtable package) 
%              and with strechable columns (from tabularx package)
% Doc: ltxtable.pdf 
% -> load afer hyperref 
\ExecuteAfterPackage{hyperref}{\usepackage{ltxtable}}

% Description: defines a single environment tabu to make all kinds of tabulars
%              It is more flexible than tabular, tabular*, tabularx and array
%              and extends the possibilities.
% Doc: tabu.pdf
\usepackage{tabu}

% tablestyles
\IfFileExists{../packages/tablestyles/tablestyles.sty}{
  \usepackage{../packages/tablestyles/tablestyles}
}{}


\EndTemplateSection{PackagesTables}

% ~~~~~~~~~~~~~~~~~~~~~~~~~~~~~~~~~~~~~~~~~~~~~~~~~~~~~~~~~~~~~~~~~~~~~~~~
% text related packages
% ~~~~~~~~~~~~~~~~~~~~~~~~~~~~~~~~~~~~~~~~~~~~~~~~~~~~~~~~~~~~~~~~~~~~~~~~

\BeginTemplateSection{PackagesText}

% description: fixes bug in ellipsis (...) 
% Doc: ellipsis.pdf
% -> load after babel
\usepackage[xspace]{ellipsis} 

%%% Text-decoration ======================================
%
% Description: commands for underlining for emphasis
% Provides: \ulin, \uuline, \sout, \xout, ...
% Doc: ulem.pdf
\usepackage[normalem]{ulem} 

% Description: commands for for emphasis
% Provides: \so, \ul, \st, ...
% Doc: soulutf8.pdf (loads soul.sty)
\usepackage{soulutf8}

% Description: enable linebreaks for URLs
% Provides: \url{}
% Doc: url.pdf
\usepackage{url}

%%% footnotes/endnotes ===================================

% Description: The footmisc package provides several different 
%              customisations of the way foonotes are represented.
%              Fixes a LaTeX bug with option 'bottom'
%
% Doc: footmisc.pdf
% Load after: setspace 
% Load before: hyperref
\ExecuteAfterPackage{setspace}{% 
%
\usepackage[%
   bottom,      % Footnotes appear always on bottom. This is necessary
                % especially when floats are used
   stable,      % Make footnotes stable in section titles
   perpage,     % Reset on each page
   %para,       % Place footnotes side by side of in one paragraph.
   %side,       % Place footnotes in the margin
   ragged,      % Use RaggedRight
   %norule,     % suppress rule above footnotes
   multiple,    % rearrange multiple footnotes intelligent in the text.
   %symbol,     % use symbols instead of numbers
]{footmisc}}

%% Description: footnotes are normally reset at each page.
%%              With this package they can be reset only at 
%%              defined headings, such as chapters.
%% Doc: chngcntr.pdf
% \usepackage{chngcntr}
% \counterwithout{footnote}{chapter}

%% Description: provides the command \tablefootnote to be used in
%%              a table or sidewaystable environment, 
%%              where \footnote will not work.
%% Doc: tablefootnote.pdf
%% Bug: does not work as expected, bug not found so far 
%% tablefootnote must be loaded after rotating
%\ExecuteAfterPackage{rotating}{%
% % and after hyperref
% \IfPackageNotLoaded{hyperref}{%
%  \ExecuteAfterPackage{hyperref}{%
%   \usepackage{tablefootnote}%
%  }%
% }{}%
%}%

%%% Doc: endnotes.pdf
% \usepackage{endnotes}

%%% References ============================================
%
% Description:  provides \vref, which is similar to \ref but 
%               adds an additional page reference, like 
%               ‘on the facing page’ or ‘on page 27’
% Doc: varioref.pdf
\usepackage{varioref} 

% Description:  enhances  the cross-referencing  features,
%               allowing the format of cross-references to be determined
%               automatically according to the “type” of cross-reference
% Doc: cleveref.pdf
% loading: must be loaded after hyperref and after varioref
\ExecuteAfterPackage{hyperref}{
% caption and cleveref incompatible in Versions before 2011/12/24
  \usepackage{cleveref}[2011/12/24]
}

%%% Lists ================================================
%
% Description: Allows the custom lists of type item, enum 
%              and description. It thereby replaces the packages
%              paralist, enumerate, mdwlist. 
% Incompatible: enumerate.
% Doc: enumitem.pdf
\IfPackageNotLoaded{enumerate}{
  \usepackage{enumitem}
}
%
\EndTemplateSection{PackagesText}

% ~~~~~~~~~~~~~~~~~~~~~~~~~~~~~~~~~~~~~~~~~~~~~~~~~~~~~~~~~~~~~~~~~~~~~~~~
% Quotes
% ~~~~~~~~~~~~~~~~~~~~~~~~~~~~~~~~~~~~~~~~~~~~~~~~~~~~~~~~~~~~~~~~~~~~~~~~
\BeginTemplateSection{PackagesQuotes}
%
% Description: Advanced features for clever quotations
% Doc: csquotes.pdf
\usepackage[%
   babel,            % the style of all quotation marks will be adapted
                     % to the document language as chosen by 'babel'
   german=quotes,    % Styles of quotes in each language
   english=british,
   french=guillemets
]{csquotes}

\EndTemplateSection{PackagesQuotes}
% ~~~~~~~~~~~~~~~~~~~~~~~~~~~~~~~~~~~~~~~~~~~~~~~~~~~~~~~~~~~~~~~~~~~~~~~~
% Citations
% ~~~~~~~~~~~~~~~~~~~~~~~~~~~~~~~~~~~~~~~~~~~~~~~~~~~~~~~~~~~~~~~~~~~~~~~~
\BeginTemplateSection{PackagesCitation}

% Description: Modern Bibliographie package with full customizability
% Doc:  biblatex.pdf
% Incompatible: ucs and every previous bibtex package
\usepackage[
  style=alphabetic, % Loads the bibliography and the citation style 
  natbib=true, % define natbib compatible cite commands
]{biblatex}  
% Other options:
%  style=numeric, % 
%  style=numeric-comp,    % [1–3, 7, 8]
%  style=numeric-verb,    % [2]; [5]; [6]
%  style=alphabetic,      % [Doe92; Doe95; Jon98]
%  style=alphabetic-verb, % [Doe92]; [Doe95]; [Jon98]
%  style=authoryear,      % Doe 1995a; Doe 1995b; Jones 1998
%  style=authoryear-comp, % Doe 1992, 1995a,b; Jones 1998
%  style=authoryear-ibid,
%  style=authoryear-icomp,
%  style=authortitle,
%  style=authortitle-comp,
%  style=authortitle-ibid,
%  style=authortitle-icomp,
%  style=authortitle-terse,
%  style=authortitle-tcomp,
%  style=authortitle-ticomp,

\EndTemplateSection{PackagesCitation}


% ~~~~~~~~~~~~~~~~~~~~~~~~~~~~~~~~~~~~~~~~~~~~~~~~~~~~~~~~~~~~~~~~~~~~~~~~
% figures, placement, floats and captions
% ~~~~~~~~~~~~~~~~~~~~~~~~~~~~~~~~~~~~~~~~~~~~~~~~~~~~~~~~~~~~~~~~~~~~~~~~
\BeginTemplateSection{PackagesFigures}

%% Description: provides new floats and enables H float modifier option
%%             (in future incompatible with Koma Script)
%% Doc: float.pdf
%% ---> replaced by floatrow package!
% \usepackage{float} 


% Description: enables typesetting a narrow float at the edge of the text,
%              and making the text wrap around it. 
% load after: float
% load before: caption
% Provides: wrapfigure and wrapfloat
% Doc: wrapfig-doc.pdf
\usepackage{wrapfig}   

% Description: place floats after the reference
% Doc: no documentation
\usepackage{flafter}

% Description: Defines a \FloatBarrier command, beyond which floats may not
%              pass; useful, for example, to ensure all floats for a section
%              appear before the next \section command.
% Doc: placeins-doc.pdf
\usepackage[
  section    % "\section" command will be redefined with "\FloatBarrier"
]{placeins}
%

%% Description: Floating figures as in wrapfloat
%%              (old LaTeX2e package from 1996)
%% Doc: floatflt.pdf
% \usepackage{floatflt}

\EndTemplateSection{PackagesFigures}

%%% Captions ============================================

\BeginTemplateSection{PackagesCaptions}


% Description: extents the float mechanism of LaTeX and
%              provides macros for precise placement of 
%              figures, tables and captions.
%              works well together with the caption pack.
% load before: caption 
% Doc: floatrow.pdf 
\usepackage{floatrow, fr-fancy}

% Description: The caption package offers customization
%              of captions in floating environments such
%              figure and table and cooperates with many 
%              other packages.
% Doc: caption.pdf (Required v3.2 or newer)
\usepackage{caption}[2011/08/06]

%% subfig ist NOT recommended, use subcaption instead
%% Incompatible: 
%% - loads package capt-of. Loading of 'capt-of' afterwards will fail therefor
%% - subcaption
%% loads: caption
%% Doc: subfig.pdf
%\usepackage{subfig} 


% Description: subcaption supports typesetting of sub-captions
%             (by using the the sub-caption feature of the caption package).
% incompatible: subfig
% Doc: subcaption.pdf
\IfPackageNotLoaded{subfig}{
  % load after caption package
  \usepackage{subcaption}[2011/08/17]
}

% Description: provides a margincap environment for putting 
%              captions into the outer document margin with 
%              either a top or bottom alignment.
% Doc: mcaption.pdf
\usepackage[
  top, %  vertical caption alignment (top, bottom)
]{mcaption}

% Description: provides two new environments, sidewaystable and sidewaysfigure,
%              and further commands to rotate content.
% Doc: rotating.pdf
\usepackage[figuresright]{rotating}

\EndTemplateSection{PackagesCaptions}


% ~~~~~~~~~~~~~~~~~~~~~~~~~~~~~~~~~~~~~~~~~~~~~~~~~~~~~~~~~~~~~~~~~~~~~~~~
% misc packages
% ~~~~~~~~~~~~~~~~~~~~~~~~~~~~~~~~~~~~~~~~~~~~~~~~~~~~~~~~~~~~~~~~~~~~~~~~
\BeginTemplateSection{PackagesMisc}

% Description: Modify printing of numbers
%              --> use siunitx instead
% Doc: numprint.pdf
% \usepackage{numprint}

% Description: adds line numbers to the main text
% Doc: ulineno
%\usepackage[
%  ,left     %  margin placment (left, right, switch, switch*)
%  ,pagewise %  Number the lines from 1 on each page (pagewise, running)
%  ,modulo   %  Print line numbers only if they are multiples of five.
%]{lineno}


\EndTemplateSection{PackagesMisc}
% ~~~~~~~~~~~~~~~~~~~~~~~~~~~~~~~~~~~~~~~~~~~~~~~~~~~~~~~~~~~~~~~~~~~~~~~~
% Index and other lists
% ~~~~~~~~~~~~~~~~~~~~~~~~~~~~~~~~~~~~~~~~~~~~~~~~~~~~~~~~~~~~~~~~~~~~~~~~

\BeginTemplateSection{PackagesIndexes}


%% Description makeindex package with shell-escape makeindex call
%% Doc: imakeidx.pdf
\usepackage{imakeidx}

% Description: print text of \index{entry} to the margin
% Doc: makeidx.pdf
% --> load only in draft mode
\IfDraft{
  \usepackage{showidx}
}

% Description: Package for glossaries, nomenclatures and acronym lists
% replaces: nomencl, acronym
% load after: hyperref!, inputenc, babel and ngerman.
\ExecuteAfterPackage{hyperref}{%
\usepackage[
  shortcuts,    % define shortcuts (\ac for acronym)
  nonumberlist, % no page numbers
  acronym,      % create a separate acronym list
  % toc,        % Add entries to toc
  section,      % add to toc on section level
  sort = standard,    % (standard, def, use)
]{glossaries}
% further styles
\usepackage{glossary-longragged}
% Create a new list of symbols
\newglossary[slg]{symbolslist}{syi}{syg}{List of Symbols}
}



\EndTemplateSection{PackagesIndexes}


% ~~~~~~~~~~~~~~~~~~~~~~~~~~~~~~~~~~~~~~~~~~~~~~~~~~~~~~~~~~~~~~~~~~~~~~~~
% verbatim packages
% ~~~~~~~~~~~~~~~~~~~~~~~~~~~~~~~~~~~~~~~~~~~~~~~~~~~~~~~~~~~~~~~~~~~~~~~~

\BeginTemplateSection{PackagesVerbatim}
%%% Doc: upquote.sty
\usepackage{upquote} % Setzt "richtige" Quotes in verbatim-Umgebung

% Description: Reimplemntation of the original verbatim enironment
% Doc: verbatim.pdf
\usepackage{verbatim} %

% Description: This package provides many facilities for reading, writing and
%              chaning the output style of verbatim code
% Doc: fancyvrb.pdf
% BUG: fails with "! Paragraph ended before \FV@BeginScanning was complete."
\usepackage{fancyvrb} 
% TODO : test

% Description: The listings package is a source code printer for LaTeX.
%              You can typeset stand alone files as well as listings with an 
%              environment.
%              If the Syntax Highlighting of the preferred  programming
%              language is not already supported, you can make your own
%              definition.
% Doc: listings.pdf
\usepackage{listings}

\EndTemplateSection{PackagesVerbatim}

% ~~~~~~~~~~~~~~~~~~~~~~~~~~~~~~~~~~~~~~~~~~~~~~~~~~~~~~~~~~~~~~~~~~~~~~~~
% fancy packages
% ~~~~~~~~~~~~~~~~~~~~~~~~~~~~~~~~~~~~~~~~~~~~~~~~~~~~~~~~~~~~~~~~~~~~~~~~

\BeginTemplateSection{PackagesFancy}

% Description: Dropping capitals
% Doc: lettrine.pdf
\usepackage{lettrine}

% Doc: boxedminipage.pdf
\usepackage{boxedminipage}

% Description: five 'fancy' boxes
% Provides: shadowbox, doublebox, ovalbox, Ovalbox
% incompatible: fancyvrb
% Doc: fancybox.pdf
% --> replaced by mdframed (take out ???)
\IfPackageNotLoaded{fancyvrb}{
  \usepackage{fancybox}
}

% Description: Create framed, shaded, or differently highlighted 
%              regions that can break across pages. 
% Doc: framed.pdf
% --> replaced by mdframed (take out ???)
\usepackage{framed}

% Description: defines new environments where the user may choose 
%              between several individual designs.
% Doc: mdframed-doc-en.pdf
\usepackage{mdframed}

\EndTemplateSection{PackagesFancy}


% ~~~~~~~~~~~~~~~~~~~~~~~~~~~~~~~~~~~~~~~~~~~~~~~~~~~~~~~~~~~~~~~~~~~~~~~~
% layout packages
% ~~~~~~~~~~~~~~~~~~~~~~~~~~~~~~~~~~~~~~~~~~~~~~~~~~~~~~~~~~~~~~~~~~~~~~~~
\BeginTemplateSection{PackagesLayout}


%%% indentation =========================================

% Description: Indent first paragraph after section header
% Doc: indentfirst.pdf
% \usepackage{indentfirst}

%%% columns =============================================

% Description: Environment for multicolumn text
% Doc: multicol.pdf
\usepackage{multicol}


%% line spacing =========================================
%
% Description: configure line spacing
% Provides: \onehalfspacing, \doublespacing
% Doc: setspace.sty
\usepackage{setspace}

%% page layout ==========================================

% Layout with 'geometry'
% Doc: geometry.pdf
% load after: hyperref
% \ExecuteAfterPackage{hyperref}{\usepackage{geometry}}

% Layout with 'typearea' 
% -> loaded automatically if geometry not loaded
% Doc: scrguide.pdf


\EndTemplateSection{PackagesLayout}

% ~~~~~~~~~~~~~~~~~~~~~~~~~~~~~~~~~~~~~~~~~~~~~~~~~~~~~~~~~~~~~~~~~~~~~~~~
% head and foot lines
% ~~~~~~~~~~~~~~~~~~~~~~~~~~~~~~~~~~~~~~~~~~~~~~~~~~~~~~~~~~~~~~~~~~~~~~~~
\BeginTemplateSection{PackagesHeadFoot}

%%% Doc: scrguide.pdf
\usepackage[%
%%% Lines
   % headtopline,
   % plainheadtopline,
   % headsepline,
   % plainheadsepline,
   % footsepline,
   % plainfootsepline,
   % footbotline,
   % plainfootbotline,
   % ilines,
   % clines,
   % olines,
% column titles (content, style)
   automark,
   % autooneside,% ignore optional argument in automark at oneside
   komastyle,
   % standardstyle,
   % markuppercase,
   % markusedcase,
   nouppercase,
]{scrpage2}


% Description: provides total number of pages (ie. page 7 of 19)
% Provides: \lastpageref{LastPage}
% load after: hyperref
% Doc: pageslts.pdf
\ExecuteAfterPackage{hyperref}{\usepackage{pageslts}}


\EndTemplateSection{PackagesHeadFoot}

% ~~~~~~~~~~~~~~~~~~~~~~~~~~~~~~~~~~~~~~~~~~~~~~~~~~~~~~~~~~~~~~~~~~~~~~~~
% layout of headings 
% ~~~~~~~~~~~~~~~~~~~~~~~~~~~~~~~~~~~~~~~~~~~~~~~~~~~~~~~~~~~~~~~~~~~~~~~~

\BeginTemplateSection{PackagesHeadings}

% Description: The titlesec package is essentially a replacement — partial or
%              total—for the LaTeX macros related with sections — namely
%              titles, headers and contents.
%%% Doc: titlesec.pdf
\usepackage{titlesec}


\EndTemplateSection{PackagesHeadings}

% ~~~~~~~~~~~~~~~~~~~~~~~~~~~~~~~~~~~~~~~~~~~~~~~~~~~~~~~~~~~~~~~~~~~~~~~~
% settings and layout of TOC
% ~~~~~~~~~~~~~~~~~~~~~~~~~~~~~~~~~~~~~~~~~~~~~~~~~~~~~~~~~~~~~~~~~~~~~~~~

\BeginTemplateSection{PackagesTOC}

% Description: apply different styles for the formating of the 
%              table of contents and lists of floats.
%%% Doc: tocstyle.pdf (Koma Script)
\usepackage[%
  tocindentauto,     % all widths at the TOCs are calculated by tocindentauto
%  tocindentmanual,  % opposite of auto
%%% 
  tocgraduated,      % standard
%  tocflat,          % no intendation, text aligned
%  tocfullflat,      % no intendation, no alignment
%%%  
  tocbreaksstrict,   % sets a lot of penalties before and after TOC entries 
                     % to avoid page break between a TOC entry and it’s parent. 
%  tocbreakscareless,% allow more page breaks.  
%%%  
%  toctextentriesindented, % unnumbered TOC entrie are indented only as wide 
%                          % as the number of numbered TOC entries of the same 
%level. 
  toctextentriesleft,% indented as if they have an empty number.
]{tocstyle}

% Description: The appendix package provides some facilities for 
%              modifying the typesetting of appendix titles.
% Doc: appendix.pdf
%\usepackage[
% ,toc   % Put a header (e.g., ‘Appendices’) into the Table of Contents
% %,page  % Puts a title  (e.g.,  ‘Appendices’) into the document at the 
%        % beginning of the appendices environment
% %,title % Adds a name (e.g., ‘Appendix’) before each appendix title in
%        % the body of the document.
% %,titletoc % Adds a name (e.g., ‘Appendix’) before each appendix listed 
%        % in the ToC
% %,header% Adds a name (e.g., ‘Appendix’) before each appendix in page headers.
%]{appendix}
%\renewcommand{\appendixtocname}{\appendixname}

\EndTemplateSection{PackagesTOC}

% ~~~~~~~~~~~~~~~~~~~~~~~~~~~~~~~~~~~~~~~~~~~~~~~~~~~~~~~~~~~~~~~~~~~~~~~~
% pdf packages
% ~~~~~~~~~~~~~~~~~~~~~~~~~~~~~~~~~~~~~~~~~~~~~~~~~~~~~~~~~~~~~~~~~~~~~~~~

\BeginTemplateSection{PackagesPDF}

% Description: Include pages from external PDF documents in LaTeX documents
% Doc: pdfpages.pdf
\usepackage{pdfpages} 

% Description: landscape orientation in PDF Format
% Doc: pdflscape.pdf
% load before footmisc
%\usepackage{pdflscape}

% Description: The microtype package provides a LaTeX interface to the  
%              micro-typographic extensions of pdfTEX: most prominently,
%              character protrusion and font expansion, furthermore
%              the adjustment of interword spacing and additional kerning.
% Provides:    Much better textformating and better typography, 
%              but at the cost of a much larger PDF file.
% Doc: microtype.pdf
\ifpdf
\usepackage{microtype}
\fi

% Description: add hyperlink support to LaTeX
% load: after almost every package!
% Doc: manual.pdf
\usepackage[
%%% Extension options
  ,backref=page       % Adds backlink text to the end of each item in the
                      % bibliography, as a list of section numbers.
                      % (section, slide, page, none)
  ,pagebackref=false  % Adds backlink text to the end of each item in the
                      % bibliography, as a list of page numbers.
  ,hyperindex=true    % Makes the page numbers of index entries into
                      % hyperlinks.
  ,hyperfootnotes=false % Makes the footnote marks into hyperlinks to the
                        % footnote text (must be false if footmisc is loaded).
%%% PDF-specific display options
  ,bookmarks=true
%%% PDF display and information options  
  ,pdfpagelabels=true % set PDF page labels
]{hyperref}

% Description: This package implements a new bookmark (outline) organization
%              for package  hyperref. In contrast to hyperref here only one 
%              LaTeX run is required.
% load: after hyperref
% Doc: bookmark.pdf
\IfNotDraft{%
  \usepackage[]{bookmark}
}

\EndTemplateSection{PackagesPDF}


% ~~~~~~~~~~~~~~~~~~~~~~~~~~~~~~~~~~~~~~~~~~~~~~~~~~~~~~~~~~~~~~~~~~~~~~~~
% additional packages 
% ~~~~~~~~~~~~~~~~~~~~~~~~~~~~~~~~~~~~~~~~~~~~~~~~~~~~~~~~~~~~~~~~~~~~~~~~

\BeginTemplateSection{PackagesAdditional}

% Description: enable hyphenation of typewriter text word (\texttt)
% Doc:  hyphenat.pdf
% Note: According to documentation the font warnings can be ignored
\usepackage[htt]{hyphenat}

\usepackage[%
  % disable,
]{todonotes}

\usepackage[NoDate]{currvita}

\EndTemplateSection{PackagesAdditional}

