% !TeX encoding=utf8
% !TeX spellcheck = en-US

\chapter{History}

The naming scheme of versions is as follows: KOMA-Script version followed by the template version. Version 3.2.0 is thus
a huge change from 3.1.0 with both compatible for version 3 of KOMA-script.

\minisec{2006/06 v2.0.0}
Initial online release of the template. It is based on KOMA-Script 2.x,
supports most modern package (at year 2006), provides most package options in the code
and a documentation of the preamble code. The basic language is German.
Additionally it provides a demo file for testing and showing the document layout.
%
\minisec{2008/12 v3.1.0} %
New release due to a rework for KOMA-Script 3.x. 
The basic design was adopted from the previous version.
Further changes mainly in terms of package updates and bug fixes.
%
\minisec{2013/06 v3.2.0} %
Initial Release of the complete reworked template with several outstanding features and changes:
\begin{itemize}
\item Complete new compilation of packages (up to date at 2013) with framework for selecting package sections. 
\item Focus on a target group of user who want to write thesis like documents.
\item Introduction of a template documentation.
\item Significant enhancements in the latex examples. 
It transformed from a simple rudimentary test and sample document to a test and example framework
with examples for every package.
\item Translation of all texts and comments into English. It targets therefor a much broader audience.
\end{itemize}
%
\minisec{2014/01 v3.2.1} %
Mainly enhancements and bug fixing. The following list is a selection:
\begin{itemize}
\item Selection of packages for the ``no room for a new \textbackslash{}write´´ problem added.  
\item Update of glossary lists handling. New file for definitions and update of \texttt{glossaries} options.
\item Added \texttt{tocstyle} to the list of used packages.
\item Added file list with date of release
\item Enabled \texttt{typearea} instead of \texttt{geometry}. This was basically a mistake in the code.
\end{itemize}
